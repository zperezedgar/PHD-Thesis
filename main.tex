\documentclass{article}

% Language setting
% Replace `english' with e.g. `spanish' to change the document language
\usepackage[english]{babel}

% Set page size and margins
% Replace `letterpaper' with `a4paper' for UK/EU standard size
\usepackage[letterpaper,top=2cm,bottom=2cm,left=3cm,right=3cm,marginparwidth=1.75cm]{geometry}

% Useful packages
\usepackage{amsmath}
\usepackage{graphicx}
\usepackage{svg}
\usepackage[colorlinks=true, allcolors=blue]{hyperref}
\usepackage{braket} % Dirac notation packet
\usepackage{comment} % to comment large sections
\usepackage{mhchem} % left superscripts
\usepackage{subcaption} % use subfigures see: https://tex.stackexchange.com/questions/200994/environment-subfigure-undefined-on-journal-submission
\captionsetup{compatibility=false} % use subfigures

% Define inverse trigonometric functions
\DeclareMathOperator{\sech}{sech}
\DeclareMathOperator{\csch}{csch}

%%%%%%%%%%%%%%%%%%%%%%%%%%%%%%%%%%%%%%%%%%%%%%%%%%%
\usepackage{titlesec} % use paragraphs with numbers
\setcounter{secnumdepth}{4}
\titleformat{\paragraph}
{\normalfont\normalsize\bfseries}{\theparagraph}{1em}{}
\titlespacing*{\paragraph}
{0pt}{3.25ex plus 1ex minus .2ex}{1.5ex plus .2ex}
%%%%%%%%%%%%%%%%%%%%%%%%%%%%%%%%%%%%%%%%%%%%%%%%%%%

\title{Gravimetry using microwaves inside an inhomogeneous magnetic field}
\author{Edgar Zuniga}

\begin{document}
\maketitle
\tableofcontents

\begin{abstract}
Your abstract.
\end{abstract}

\section{Theoretical Framework}

\subsection{Condition for Atomic Interference Inside an Inhomogeneous Magnetic Field}
\subsubsection{Semi-classical Picture}

In this section, we review the condition needed to generate atomic interference in a gravimeter within a magnetic gradient. Let's consider an atom with total angular momentum number $F$. We will consider transitions between two hyperfine levels with different magnetic quantum number $m_{F}$. The only requirement is that at least one of the levels has to be sensible to the magnetic field and that the response of each level to the magnetic field has to be different\footnote{For instance, we can choose to work with the stretched states with quantum magnetic number $m_{F}=\pm F$.}. Let's denote the ground state of our system as $\ket{1}$, and the excited state as $\ket{2}$.

The atom is immersed inside a magnetic field of the following form

\begin{equation}\label{magnetic_field}
\textbf{B} = \eta z \hat{\textbf{z}}\
\end{equation}

where z is the position and $\eta$ is a constant with units of Tesla per meter. The atom will feel a force given by the gradient of the magnetic energy caused by the coupling of the total magnetic moment of the atom $\mu_{z}$ along the z-direction and the magnetic field\footnote{It is important to comment out that in the rest frame of the atom, we would see that there is a magnetic field that changes with time. Therefore, according to Faraday´s law, this changing magnetic field would induce an electric field that would couple with the atom's electric dipole moment giving rise to an electric energy term that would try to hold back the atom from falling down and would need to be considered. Fortunately, we can choose to work with atomic states such that the matrix elements of the position operator vanish at first order.}. The total magnetic moment of the atom depends on its internal state, i.e., the hyperfine level under consideration. For this calculation, we will suppose that the two-level system of our atom is such that the magnetic force for each level is be given by

\begin{equation}\label{magnetic_force}
\textbf{F} = - \nabla (\pm \mu_{z} \eta z ) = \mp \mu_{z} \eta \hat{\textbf{z}},\
\end{equation}

i.e., the atom's magnetic moment for the ground state is anti-parallel to the magnetic moment for the excited state.
This magnetic force will cause an acceleration in the atom which will be different for each level. Let $a_B$ be the acceleration due to the magnetic gradient for the hyperfine state $\ket{1}$ and $-a_B$ the acceleration for the hyperfine state $\ket{2}$. Thus, the effective accelerations for state $\ket{1}$ and state $\ket{2}$ are respectively

\begin{equation}\label{a1}
a_{+} = g + a_B
\end{equation}
\begin{equation}\label{a2}
a_{-} = g - a_B
\end{equation}

where g is the local acceleration due to gravity (taken to be positive), and $a_B$ is given by

\begin{equation}\label{am}
a_B = \mu_{z} \eta / m
\end{equation}

where $m$ is the mass of the atom under consideration. Note that the subscript $+$ indicates that we are considering the state $\ket{1}$ while the subscript $-$ indicates that we are considering the state $\ket{2}$.

In order to produce interference between these two states, we have to apply a $\frac{\pi}{2}$-pulse followed by a sequence of $\pi$-pulses and finally another $\frac{\pi}{2}$-pulse. The question that arises is, how many $\pi$-pulses do we need to apply in order to produce interference? To answer this question, we can picture a semi-classical approach where the interference condition is that the final position and the final velocity must coincide for both states.

\paragraph{Sequence of pulses $\frac{\pi}{2} - \pi - \frac{\pi}{2}$}
Let's study the classic sequence of pulses used to perform gravimetry and let's see if it can be used to obtain a signal for gravimetry in the presence of the magnetic force Eq. \ref{magnetic_force}.

Let $z_0$ and $v_0$ be the initial position and velocity respectively. The final velocity for each state is given respectively by the classical equation

\begin{equation}\label{v1_sequence_classic}
v_{1} = v_{0} + a_{+} T_{1} + a_{-} T_{2}
\end{equation}
\begin{equation}\label{v2_sequence_classic}
v_{2} = v_{0} + a_{-} T_{1} + a_{+} T_{2}
\end{equation}

where $T_{1}$ is the time elapsed between the first $\frac{\pi}{2}$-pulse and the $\pi$-pulse, and $T_{2}$ is the time elapsed between the $\pi$-pulse and the second $\frac{\pi}{2}$ pulse.

On the other hand, the final position for each state is given by the classical equation of motion

\begin{equation}\label{z1_sequence_classic}
z_{1} = (z_{0} + v_{0} T_{1} + \frac{a_{+} T^{2}_{1}}{2}) + (v_{0}+a_{+}T_{1})T_{2} + \frac{a_{-} T^{2}_{2}}{2}
\end{equation}
\begin{equation}\label{z2_sequence_classic}
z_{2} = (z_{0} + v_{0} T_{1} + \frac{a_{-} T^{2}_{1}}{2}) + (v_{0}+a_{-}T_{1})T_{2} + \frac{a_{+} T^{2}_{2}}{2}
\end{equation}

By using Eqs. \ref{v1_sequence_classic}-\ref{v2_sequence_classic}, and by demanding that the final velocities must be equal, we get that in order to produce interference, the acceleration $a_{+}$ must equal $a_{-}$ or alternatively, $T_{1}$ must equal $T_{2}$. Since, by assumption, $a_{+}$ and $a_{-}$ cannot be the same, then, the only acceptable condition is $T_{1}=T_{2}$. By accepting this condition on the duration of the pulses, and by using Eqs. \ref{z1_sequence_classic}-\ref{z2_sequence_classic} to demand that the final positions must be the same, we recover the unacceptable condition $a_{+}=a_{-}$. Therefore, we cannot produce interference in the presence of a magnetic force by applying just one $\pi$-pulse between the $\frac{\pi}{2}$ pulses.

\paragraph{Sequence of pulses $\frac{\pi}{2} - \pi - \pi - \frac{\pi}{2}$}

Let's study the sequence of pulses with an extra $\pi$-pulse before the last $\frac{\pi}{2}$-pulse. In this case, the equations for the final velocities are

\begin{equation}\label{final_v1}
v_{1} = v_{0} + a_{+} T_{1} + a_{-} T_{2} + a_{+} T_{3}
\end{equation}
\begin{equation}\label{final_v2}
v_{2} = v_{0} + a_{-} T_{1} + a_{+} T_{2} + a_{-} T_{3}
\end{equation}

Where $T_{3}$ is the time elapsed between the second $\pi$-pulse and the second $\frac{\pi}{2}$-pulse. Likewise, the equations for the final positions are

\begin{equation}\label{final_z1}
z_{1} = [(z_{0} + v_{0} T_{1} + \frac{a_{+} T^{2}_{1}}{2}) + (v_{0} + a_{+}T_{1})T_{2} + \frac{a_{-} T^{2}_{2}}{2}] + (v_{0}+a_{+}T_{1} + a_{-}T_{2})T_{3} + \frac{a_{+} T^{2}_{3}}{2}
\end{equation}
\begin{equation}\label{final_z2}
z_{2} = [(z_{0} + v_{0} T_{1} + \frac{a_{-} T^{2}_{1}}{2}) + (v_{0} + a_{-}T_{1})T_{2} + \frac{a_{+} T^{2}_{2}}{2}] + (v_{0}+a_{-}T_{1} + a_{+}T_{2})T_{3} + \frac{a_{-} T^{2}_{3}}{2}
\end{equation}

By using Eqs. \ref{final_v1} and \ref{final_v2}, we get an equation that relates the periods between pulses as

\begin{equation}\label{times_relation}
T_{a} = 1 + T_{b}
\end{equation}

where we have defined $T_{a} \equiv \frac{T_{2}}{T_{1}}$ and $T_{b} \equiv \frac{T_{3}}{T_{1}}$. Now, by using Eqs. \ref{final_z1}, \ref{final_z2}, and \ref{times_relation}, we get the condition for the times elapsed between each pulse

\begin{equation}
\begin{aligned}
T_{2} = 2T_{1} \\
T_{1} = T_{3}
\end{aligned}
\end{equation}

These equations hold independently of the values for $a_{+}$ and $a_{-}$, as can be shown by substituting them into Eqs. \ref{final_v1} and \ref{final_v2}. Therefore, the condition to get interference in the presence of a magnetic force is to apply a 
$\frac{\pi}{2}$-pulse followed by a $\pi$-pulse after a time $T$ has elapsed, then apply another $\pi$-pulse after a time $2T$ has elapsed, and finally, apply a $\frac{\pi}{2}$-pulse after a time $T$ elapsed, i.e., the sequence of pulses is

\begin{equation}\label{pulses}
\boldsymbol{\pi/2} \xrightarrow[]{T} \boldsymbol{\pi} \xrightarrow[]{2T} \boldsymbol{\pi} \xrightarrow[]{T} \boldsymbol{\pi/2}
\end{equation}

After the first $\frac{\pi}{2}$-pulse is applied, a superposition of states is created. In this semi-classical picture, each state moves independently with a different acceleration, that depends on the hyperfine levels under consideration and the magnitude of the gradient field at the position where each the state is located. Every time that a $\pi$-pulse is applied, the velocity of each state changes as shown in Fig. \ref{velocity_graph}. The final velocity for both states is the same as was required by the interference condition. Similarly, the final position for both states is the same just before the last $\frac{\pi}{2}$-pulse is applied in order to produce interference as can be seen in Fig. \ref{position_graph}. Interestingly, the position curves never cross each other during the lifetime of the superposition, unlike the velocity curves which cross each other once.

\begin{figure}
    \centering
    \begin{subfigure}{1\textwidth}
        \centering
        \includegraphics[width=0.8\textwidth]{velocidad.png}
        \caption{Velocity-time graph.}
        \label{velocity_graph}
    \end{subfigure}
    \hfill
    \begin{subfigure}{1\textwidth}
        \centering
        \includegraphics[width=0.8\textwidth]{posicion.png}
        \caption{Position-time graph.}
        \label{position_graph}
    \end{subfigure}
    \caption{Velocity and position vs. time graphs for each state after the superposition of hyperfine levels is created and just before they are made interfere. In these graphs $g=9.8m/s^2$, the mass $m$ is taken to be $87*1.6*10^{-27} Kg$ (approximately the mass of $\ce{^{87}_{}Rb}$), $\mu_{z}$ is taken to be the Bohr magneton with value $9.27*10^{-24} J/T$, $\eta$ has a value of $0.05 T/m$, and $T=100ms$. 
    The hyperfine state $\ket{1}$ has a quantum magnetic number of $m_{F}=F$ whereas the state $\ket{2}$ has a value of $m_{F}=-F$.}
    \label{velocity_position_graphs}
\end{figure}

\paragraph{Calculation of the Signal for Gravimetry}
Now that we have settled down the sequence of pulses needed to produce interference, we are in a position to compute the expected signal for gravimetry. In the semi-classical picture, each state of the superposition follows a different path and consequently accumulates a different phase. Therefore, to obtain the gravimetry signal, we need to compute the phase accumulated by each state during the lifetime of the superposition and then compute the difference between these two phases. 

The phase accumulated for each path is given by

\begin{equation}
\Phi = \int_{t_{0}}^{t} \omega(\ce{^{}_{}t{'}_{}})d\ce{^{}_{}t{'}_{}}
\end{equation}

where the angular frequency is proportional to the energy according to the Planck relation\footnote{The reader may be wondering why this semi-classical relation holds instead of a relation with a different proportionality factor of the form $E = \alpha \hbar \omega$ where $\alpha$ is a constant. The reason is that this relation comes from the time dependent solution of the Schrodinger equation when the Hamiltonian is independent of time, i.e., $e^{-i E t/\hbar}$, so it is legal to use it here.}

\begin{equation}\label{planck_relation}
E = \hbar \omega, 
\end{equation}

and the energy is given by the sum of the magnetic energy\footnote{In a quantum computation, we would have had to use the expectation value of the angular momentum $<\mu_{z}>$ with respect to the levels of interest, i.e., the energy would have been computed as the expectation value of the corresponding Hamiltonian.} (Eq. \ref{magnetic_force}) and the energy due to the gravity field, i.e.,

\begin{equation}\label{classic_energy}
E = \pm \mu_{z} B_{z} + mgz = (\pm \mu_{z} \eta + mg)z = C_{\pm} z
\end{equation}

where we have defined the constant

\begin{equation}\label{c_definition}
C_{\pm} \equiv \pm \mu_{z} \eta + mg = m a_{\pm}
\end{equation}

with units of force and with $a_{\pm}$ defined as in the Eqs. \ref{a1} and \ref{a2}. Notice that we have defined the constant $g$ to be positive and we are using a coordinate system with the z-axis pointing upwards such that the atom is initially located at $z=0$, where the initial gravitational potential energy is zero, and as it falls down, the position coordinate becomes negative so the gravitational potential energy becomes negative.

Now, we can proceed to compute the accumulated phase given by

\begin{equation}\label{semi_classical_phase}
\Phi = \frac{C_{\pm}}{\hbar} \int_{t_{0}}^{t} z(\ce{^{}_{}t{'}_{}})d\ce{^{}_{}t{'}_{}}
\end{equation}

Each state's path is composed by three intervals $T_{1}=T$, $T_{2}=2T$, and $T_{3}=T$ (see Eq. \ref{pulses}). Thereby, for the first interval, the phase accumulated by each state will be respectively

\begin{equation}\label{phi1t1}
\Phi_{1, T_{1}} = \frac{C_{-}}{\hbar} \int_{0}^{T} (z_{0}+v_{0}t+a_{+} \frac{t^{2}}{2})dt
\end{equation}

\begin{equation}
\Phi_{2, T_{1}} = \frac{C_{+}}{\hbar} \int_{0}^{T} (z_{0}+v_{0}t+a_{-} \frac{t^{2}}{2})dt
\end{equation}

where $z_{0}$ and $v_{0}$ are respectively the initial position and velocity,
and the accelerations $a_{+}$ and $a_{-}$ are given by Eqs. \ref{a1} and \ref{a2}. The phase difference accumulated during the first interval is

\begin{equation}
\Delta \Phi_{T_{1}} = \Phi_{1, T_{1}} - \Phi_{2, T_{1}}
\end{equation}

Likewise, the phase accumulated during the second interval for each state is

\begin{equation}
\Phi_{1, T_{2}} = \frac{C_{+}}{\hbar} \int_{0}^{2T} [(z_{0}+v_{0}T+a_{+} \frac{T^{2}}{2}) + (v_{0}+a_{+}T)t + a_{-} \frac{t^{2}}{2}]dt
\end{equation}

\begin{equation}
\Phi_{2, T_{2}} = \frac{C_{-}}{\hbar} \int_{0}^{2T} [(z_{0}+v_{0}T+a_{-} \frac{T^{2}}{2}) + (v_{0}+a_{-}T)t + a_{+} \frac{t^{2}}{2}]dt
\end{equation}

The corresponding phase difference for this interval is

\begin{equation}
\Delta \Phi_{T_{2}} = \Phi_{1, T_{2}} - \Phi_{2, T_{2}}
\end{equation}

Finally, for the last interval, the phase accumulated by each state is

\begin{equation}
\Phi_{1, T_{3}} = \frac{C_{-}}{\hbar} \int_{0}^{T} [((z_{0}+v_{0}T+a_{+} \frac{T^{2}}{2}) + (v_{0}+a_{+}T)(2T) + a_{-} \frac{(2T)^{2}}{2}) + (v_{0}+a_{+}T + a_{-}(2T))t + a_{+} \frac{t^{2}}{2}]dt
\end{equation}

\begin{equation}
\Phi_{2, T_{3}} = \frac{C_{+}}{\hbar} \int_{0}^{T} [((z_{0}+v_{0}T+a_{-} \frac{T^{2}}{2}) + (v_{0}+a_{-}T)(2T) + a_{+} \frac{(2T)^{2}}{2}) + (v_{0}+a_{-}T + a_{+}(2T))t + a_{-} \frac{t^{2}}{2}]dt
\end{equation}

The phase difference accumulated during the last interval is

\begin{equation}\label{delphi3}
\Delta \Phi_{T_{3}} = \Phi_{1, T_{3}} - \Phi_{2, T_{3}}
\end{equation}

Therefore, by using Eqs.(\ref{a1}-\ref{a2} , and \ref{phi1t1}-\ref{delphi3}), the total phase difference accumulated during the three intervals is

\begin{equation}
\Delta \Phi = \Delta \Phi_{T_{1}} + \Delta \Phi_{T_{2}} + \Delta \Phi_{T_{3}} = 8 \frac{mga_{m}T^3}{\hbar}
\end{equation}

\begin{equation}\label{gravimetry_signal}
\Delta \Phi = 8 \frac{\mu_{z} \eta}{\hbar} g T^3
\end{equation}

where in the last equation we have used the definition of $a_B$ (Eq. \ref{am}). 

The Eq. \ref{gravimetry_signal} is the signal for gravimetry, which is proportional to $T^3$ and also proportional to the magnetic field via the constant $\eta$, these are the parameters we can adjust within a gravimetric experiment in order to increase the precision of the measurement. In Fig.\ref{phase_graph}, we can see the strong of the gravimetry signal for typical values of the constants in Eq. \ref{gravimetry_signal}. 

Finally, we can compare this result with the signal measured in traditional gravimetry that uses stimulated Raman transitions to induce state transitions and changes in momentum \cite{Peters_2001}

\begin{equation}\label{traditional_gravimetry_signal}
\Delta \Phi = k_{eff} g T^2 ,
\end{equation}

where $\hbar k_{eff}$ is the effective momentum gained by the atom after the Raman transition has finished.

\begin{figure}
\centering
\includegraphics[width=0.9\textwidth]{fase.png}
\caption{Gravimetry signal for an atom falling down inside a linear magnetic field. This graph was obtained by using the same constant values as described in Fig.\ref{velocity_position_graphs}.}
\label{phase_graph}
\end{figure}

\subsubsection{Quantum Picture}
The calculations above were made by using a semi-classical picture to represent the superposition of states. Nevertheless, a correct calculation of the signal for gravimetry demands solving the Schrodinger equation for the evolution of each state in the superposition. 

In the following, we will consider a coherent wave packet of alkali-metal atoms with quantized center-of-mass motion along the z-axis, subject to a constant gravitational field and interacting with a classical magnetic field of the form given by Eq. \ref{magnetic_field}. The wave packet is given at time $t=0$ by

\begin{equation}\label{wave_packet}
\psi(z,0) = 
\left[\frac{1}{2 \pi (\Delta Z(0))^2} \right]^{1/4} \exp \left[-\frac{1}{4} \left[\frac{z-z_{0}}{\Delta Z(0)}\right]^{2}  + i \frac{p_{0}}{\hbar}z \right]
\end{equation}

where $z$ is the position coordinate, $\Delta Z(0)$ is the width of the wave packet centered at $z_{0}$, and $\frac{p_{0}}{\hbar}$ is the wave number of the wave packet. 

Before continuing, we introduce the non-dimensional position $x \equiv \kappa z$, and define the non-dimensional representation of this wave packet as

\begin{equation}\label{wave_packet_non_dimensional}
\phi(x,0) \equiv \frac{1}{\sqrt{\kappa}} \psi(\frac{x}{\kappa}, 0)
\end{equation}

where we have defined the inverse of the characteristic length of the system as

\begin{equation}\label{kappa}
\kappa \equiv (g_{s}\frac{\mu_{B}}{\hbar} + g_{n}\frac{\mu_{n}}{\hbar}) \frac{\eta}{\Delta W} > 0
\end{equation}

where $g_{s}$ is the electron spin $g$ factor, $g_{n}$ is the nuclear $g$ factor, $\mu_{B}$ is the Bohr magneton, $\mu_{n}$ is the nuclear magneton, $\hbar \Delta W$ is the field-free ground-state hyperfine splitting energy, and $\eta$ is the proportionality constant of the magnetic field (Eq. \ref{magnetic_field}).

The time-evolution of the non-dimensional wave packet (Eq. \ref{wave_packet_non_dimensional}) is given by\footnote{The wave packet in the Eq. \ref{wave_packet_non_dimensional_evolution} is already normalized in position for all times.} (see \cite{Castanos2014} for additional details)

\begin{equation}\label{wave_packet_non_dimensional_evolution}
\phi(x,\tau) = 
\left[\frac{1}{2 \pi (\sigma_{x}(\tau))^2} \right]^{1/4} \exp \left[-\frac{1}{4} \left[\frac{x-\xi(\tau)}{\sigma_{x}(\tau)}\right]^{2}  + i\Theta_{p}(x, \tau) - \frac{i}{2}\arctan\left[\frac{\epsilon \tau}{(\sigma_{x}(0))^{2}}\right] \right]
\end{equation}

where $\tau\equiv \Delta W t$ is the non-dimensional time, and we have defined 

\begin{multline}\label{theta_p}
\Theta_{p}(x,\tau) = -\tau \bigg(q_{1} + \frac{\epsilon}{3} q_{2}^{2} \tau^{2}\bigg) + \left[\frac{\sigma_{x}(0)}{\sigma_{x}(\tau)} \right]^{2} (\rho_{0} - q_{2} \tau)(x-\rho_{0} \epsilon \tau) \\
+ \frac{\epsilon \tau}{4 [\sigma_{x}(\tau)\sigma_{x}(0)]^{2}} \left[(x-x_{0})^{2} + 4\rho_{0} x_{0} \epsilon \tau -2q_{2} \epsilon \tau^{2} \bigg(x+x_{0}- \frac{q_{2}}{2} \epsilon \tau^{2} \bigg)\right]
\end{multline}

where we have introduced the following non-dimensional quantities

\begin{equation}\label{non_dimensional_definitions}
\sigma_{x}(0) = \kappa \Delta Z(0)\mathrm{,}\quad \rho_{0}=\frac{p_{0}}{\hbar \kappa} \mathrm{,}\quad 
x_{0}=\kappa z_{0} \mathrm{,}\quad 
\sigma_{x}(\tau) = \sqrt{[\sigma_{x}(0)]^{2} + \left[\frac{\epsilon \tau}{\sigma_{x}(0)} \right]^{2}}
\end{equation}

were $\epsilon$ is given by 

\begin{equation}\label{epsilon}
\epsilon = \frac{\hbar \kappa^{2}}{2 M \Delta W}
\end{equation}

where $M$ is the mass of the atom. Also, we have defined

\begin{equation}\label{q1_q2}
q_{1} = \frac{1}{2} \mathrm{,}\quad q_{2} = \bigg(\frac{M g}{\hbar \Delta W \kappa} \pm  \gamma_{1} \bigg)
\end{equation}

where $g_{0}$ is the acceleration of gravity, and 

\begin{equation}\label{gamma_1}
\gamma_{1} = \frac{g_{s}-2 I g_{n} \frac{m_{e}}{m_{p}}}{2(g_{s}+g_{n}\frac{m_{e}}{m_{p}})}
\end{equation}

where $m_{e}$ and $m_{p}$ are the electron mass and the proton mass respectively, and $I$ is the nuclear spin (assumed to be $I\ \ge 1/2$).
We will see lately when comparing with the time-evolution of the free wave packet that $\Delta W q_{1}$ corresponds to the angular frequency and therefore, it is related with the energy of the wave packet.


Additionally, we have defined the following non-dimensional quantity

\begin{equation}\label{xi}
\xi(\tau) = x_{0} + 2 \rho_{0} \epsilon \tau - q_{2}\epsilon \tau^{2}
\end{equation}

The above equation describes the motion of the center of the wave packet with respect to the non-dimensional time $\tau$ as can be seen by using Eqs. \ref{non_dimensional_definitions} and \ref{epsilon} to cast the equation into its dimensional form as follows

\begin{equation}\label{xi_dimensional}
z(t) = \frac{\xi}{\kappa} = z_{0} +  \beta_{0} t +  \frac{\alpha t^{2}}{2}
\end{equation}

where the initial velocity is

\begin{equation}
\beta_{0} = \frac{p_{0}}{M}
\end{equation}

and the acceleration is given by

\begin{equation}\label{q2_a}
\alpha = -q_{2} \frac{ \hbar \kappa \Delta W}{M}
\end{equation}

Therefore, it is convenient to re-write Eq. \ref{xi} as

\begin{equation}\label{xi_eq_motion}
\xi(\tau) = x_{0} + v_{0} \tau + \frac{a \tau^{2}}{2}
\end{equation}

where $v_{0}$ is the non-dimensional initial velocity and $a$ is the non-dimensional acceleration given by

\begin{equation}\label{xi_defs_eq_motion}
v_{0} = 2\rho_{0} \epsilon \mathrm{,}\quad a = -2 q_{2} \epsilon
\end{equation}

Therefore, the non-dimensional velocity of the center of the wave packet will be given by the derivative of Eq. \ref{xi} with respect to $\tau$, i.e.,

\begin{equation}\label{xi_derivative}
\frac{d\xi}{d\tau} = \xi^{\prime}(\tau) =  2 \rho_{0} \epsilon - 2 q_{2}\epsilon \tau = v_{0} + a \tau
\end{equation}

For that reason, Eqs. \ref{xi} and \ref{xi_derivative} are the equations of motion of the center of the (non-dimensional) wave packet. Additionally, we notice that the information about the acceleration of the wave-packet is contained in the parameter $q_{2}$. This parameter is obviously related to the accelerations defined in Eqs. \ref{a1} and \ref{a2} and by consequence with the constant $C_{\pm}$ defined in Eq. \ref{c_definition}.

\paragraph{Calculation of the Signal for Gravimetry in Position Space}

Now, we proceed to compute the signal for gravimetry by using Eq. \ref{wave_packet_non_dimensional_evolution}. In order to do so, we identify the phase of the wave packet

\begin{equation}
\varphi(x, \tau) = \Theta_{p}(x, \tau) - \frac{1}{2}\arctan\left[\frac{\epsilon \tau}{(\sigma_{x}(0))^{2}}\right]
\end{equation}

or more explicitly, using Eq. \ref{theta_p}

\begin{multline}\label{quantum_phase}
\varphi(x, \tau) = -\tau \bigg(q_{1} + \frac{\epsilon}{3} q_{2}^{2} \tau^{2}\bigg) + \left[\frac{\sigma_{x}(0)}{\sigma_{x}(\tau)} \right]^{2} (\rho_{0} - q_{2} \tau)(x-\rho_{0} \epsilon \tau) \\
+ \frac{\epsilon \tau}{4 [\sigma_{x}(\tau)\sigma_{x}(0)]^{2}} \left[(x-x_{0})^{2} + 4\rho_{0} x_{0} \epsilon \tau -2q_{2} \epsilon \tau^{2} \bigg(x+x_{0}- \frac{q_{2}}{2} \epsilon \tau^{2} \bigg)\right] 
- \frac{1}{2}\arctan\left[\frac{\epsilon \tau}{(\sigma_{x}(0))^{2}}\right]
\end{multline}

The above equation contains several corrections to the phase computed by using Eq. \ref{semi_classical_phase} which was obtained by using a semi-classical approach. The last term in Eq. \ref{quantum_phase} does not contribute to the phase difference, as we will see in the next section\footnote{The last term does not have any dependence on the initial velocity, initial position, and/or $q_{2}$, therefore, its contribution to the phase difference will be zero.}, and we can ignore it. The semi-classical result (Eq. \ref{gravimetry_signal}) can be reproduced by using the following approximation

\begin{equation}\label{sigma_zero_approx}
\sigma_{x}(0) \gg \epsilon \tau,
\end{equation}

in the definition of $\sigma_{x}(\tau)$ which we rewrite here

\begin{equation*}
\sigma_{x}(\tau) = \sqrt{[\sigma_{x}(0)]^{2} + \left[\frac{\epsilon \tau}{\sigma_{x}(0)} \right]^{2}},
\end{equation*}

therefore, using this approximation we can write

\begin{equation*}
\sigma_{x}(\tau) \approx \sigma_{x}(0),
\end{equation*}

so we will have the following

\begin{equation}\label{sigma_quotient_approx}
\left[\frac{\sigma_{x}(0)}{\sigma_{x}(\tau)} \right]^{2} \approx 1 \mathrm{,}\quad \frac{\epsilon \tau}{ [\sigma_{x}(\tau)\sigma_{x}(0)]^{2}} \approx 0.
\end{equation}

The physical interpretation of this approximation (Eq. \ref{sigma_zero_approx}) is that the widening of the wave packet during the experiment is negligible. In other words, the waist of the wave packet remains approximately constant.

Therefore, using the approximation in Eq. \ref{sigma_zero_approx} (and discarding the term that does not contribute at all) allows us to approximate the phase (Eq. \ref{quantum_phase}) as

\begin{equation}\label{approx_quantum_phase}
\varphi_{\pm}(x, \tau) \approx -\tau \bigg(q_{1} + \frac{\epsilon}{3} q_{2_{\pm}}^{2} \tau^{2}\bigg) + (\rho_{0} - q_{2_{\pm}} \tau)(x-\rho_{0} \epsilon \tau)
\end{equation}

where we have used a sub-index to indicate whether we are considering $q_{2}$ with the plus or minus sign in front of the constant $\gamma_{1}$. 

Before diving deep into the computation of the phase difference, let's analyze the above equation. The second term is the product of two differences. The first term is 

\begin{equation}\label{momentum_change}
(\rho_{0} - q_{2_{\pm}} \tau),
\end{equation}

according to Eq. \ref{non_dimensional_definitions}, this term accounts for the increase in momentum of the wave-packet, therefore, it can be interpreted as the difference between the initial momentum $\rho_{0}$ and the final momentum $q_{2_{\pm}} \tau$ acquired due to the acceleration. In the same way, the second term is

\begin{equation}\label{position_change}
(x-\rho_{0} \epsilon \tau),
\end{equation}

according to Eq. \ref{xi_defs_eq_motion}, the product $\rho_{0} \epsilon$ represents the initial velocity of the wave packet, therefore, the above term represents the difference between the non-dimensional position $x$ and half of the non-dimensional distance acquired due to the initial velocity of the wave packet.

Now, we are in position to compute the signal for gravimetry using Eq. \ref{approx_quantum_phase}, and the sequence of pulses shown in Eq. \ref{pulses}. Also, according to Eq. \ref{approx_quantum_phase}, the phase obtained will depend on the non-dimensional position $x$, thereby, all the contributions to the phase have to be computed with respect to the same $x$ which we will call\footnote{$x_{d}$ can be, for instance, the position of the detector at the end of the experiment.} $x_{d}$.

Let's analyze the first interval. The initial momentum and velocity of the wave packet are respectively $\rho_{0}$ and $v_{0}$. Thereby, the phase acquired during the first interval will be

\begin{equation}\label{approx_quantum_phase_1}
\varphi_{\pm, \tau_{1}}(x) \approx -\tau \bigg(q_{1} + \frac{\epsilon}{3} q_{2_{\pm}}^{2} \tau^{2}\bigg) + (\rho_{0} - q_{2_{\pm}} \tau)(x-\frac{v_{0} \tau}{2})
\end{equation}

where we have used the definition of $v_{0}$, Eq. \ref{xi_defs_eq_motion}.
For the second interval (with period $2\tau$), the initial momentum of the wave-packet will be $(\rho_{0} - q_{2_{\pm}} \tau)$ and the initial velocity will be $(v_{0}+a_{\pm}\tau)$, where the acceleration $a_{\pm}$ was defined in Eq. \ref{xi_defs_eq_motion} and the sub-index indicates the sign used in $q_{2}$ (Eq. \ref{q1_q2}). Therefore, the phase accumulated during the second interval will be given by

\begin{equation}\label{approx_quantum_phase_2}
\varphi_{\pm, \tau_{2}}(x) \approx -(2\tau) \bigg(q_{1} + \frac{\epsilon}{3} q_{2_{\mp}}^{2} (2\tau)^{2}\bigg) + ((\rho_{0} - q_{2_{\pm}} \tau)-q_{2_{\mp}} (2\tau))(x-\frac{(v_{0}+a_{\pm}\tau) (2\tau)}{2})
\end{equation}

where we have used a sub-index in $a$ to indicate the sign of $\gamma_{1}$ in the definition of $q_{2}$.
Similarly, the initial momentum of the wave-packet for the third interval will be $(\rho_{0} - q_{2_{\pm}} \tau-2q_{2_{\mp}}\tau)$ and the initial velocity will be $(v_{0}+a_{\pm}\tau + 2 a_{\mp}\tau)$, thus the phase acquired during this interval will be given by

\begin{equation}\label{approx_quantum_phase_3}
\varphi_{\pm, \tau_{3}}(x) \approx -\tau \bigg(q_{1} + \frac{\epsilon}{3} q_{2_{\pm}}^{2} \tau^{2}\bigg) + ((\rho_{0} - q_{2_{\pm}} \tau-2q_{2_{\mp}}\tau)-q_{2_{\pm}} \tau)(x-\frac{(v_{0}+a_{\pm}\tau + 2a_{\mp}\tau) \tau}{2})
\end{equation}

Therefore, the phase difference accumulated during the first interval, and evaluated at the position $x_{d}$ will be given by

\begin{equation}
\Delta \Phi_{\tau_{1}}(x_{d}) = \varphi_{+,\tau_{1}}(x_{d}) - \varphi_{-,\tau_{1}}(x_{d})
\end{equation}

whereas for the second interval, evaluated at $x_{d}$, we have

\begin{equation}
\Delta \Phi_{\tau_{2}}(x_{d}) = \varphi_{-,\tau_{2}}(x_{d}) - \varphi_{+,\tau_{2}}(x_{d})
\end{equation}

and finally, for the last interval, evaluated at $x_{d}$, we have

\begin{equation}
\Delta \Phi_{\tau_{3}}(x_{d}) = \varphi_{+,\tau_{3}}(x_{d}) - \varphi_{-,\tau_{3}}(x_{d})
\end{equation}

Thereby, the total phase difference accumulated during the three intervals at $x_{d}$ is given by 

\begin{equation}\label{pre_total_quantum_phase}
\Delta \Phi (x_{d}) = \Delta \Phi_{\tau_{1}}(x_{d}) + \Delta \Phi_{\tau_{2}}(x_{d}) + \Delta \Phi_{\tau_{3}}(x_{d})
\end{equation}

Now, before substituting numerical values, we will make additional approximations. Firstly, we need an approximation for $\kappa$ defined in Eq. \ref{kappa}. The Bohr magneton, and the nuclear magneton are defined as

\begin{equation}
\mu_{B} = \frac{\hbar e}{2 m_{e}} \mathrm{,}\quad \mu_{n} = \frac{\hbar e}{2 m_{p}}
\end{equation}

where $m_{e}$ and $m_{p}$ are the electron and proton mass respectively. However, the proton mass is several orders of magnitude larger than the electron mass, therefore, we can use the following approximation

\begin{equation}\label{kappa_approx}
\kappa \approx g_{s}\frac{\mu_{B}}{\hbar} \frac{\eta}{\Delta W}
\end{equation}

Additionally, by considering the alkali atoms to be $\ce{^{87}_{}Rb}$ atoms. We have the following values for this atom \cite{KAUSHALSK1970,Bunge1993} 

\begin{equation}
g_{s} \approx 2.002 \mathrm{,}\quad g_{n} \frac{m_{e}}{m_{p}} \approx 2.002 \mathrm{,}\quad I = \frac{3}{2}
\end{equation}

thereby, we can approximate $\gamma_{1}$ defined in Eq. \ref{gamma_1} as

\begin{equation}\label{gamma_1_approx}
\gamma_{1} \approx 1/2
\end{equation}

Finally, by using Eqs.(\ref{approx_quantum_phase_1}-\ref{pre_total_quantum_phase}), the definitions Eqs.(\ref{non_dimensional_definitions}-\ref{gamma_1}) alongside the approximations in Eqs. \ref{kappa_approx}, \ref{gamma_1_approx}, we obtain

\begin{equation}
\Delta \Phi_{\tau_{1}}(x_{d}) = -x_{d}\tau + \frac{\eta \mu_{B} \tau^{2} p_{0}}{M \hbar \Delta W^{2}} - \frac{2 \eta \mu_{B} \tau^{3} g}{3 \hbar \Delta W^{3}}
\end{equation}

\begin{equation}
\Delta \Phi_{\tau_{2}}(x_{d}) = x_{d}\tau + \frac{4 \eta \mu_{B} \tau^{3} g}{3 \hbar \Delta W^{3}}
\end{equation}

\begin{equation}
\Delta \Phi_{\tau_{3}}(x_{d}) = -\frac{\eta \mu_{B} \tau^{2} p_{0}}{M \hbar \Delta W^{2}} + \frac{10 \eta \mu_{B} \tau^{3} g}{3 \hbar \Delta W^{3}}
\end{equation}

\begin{equation}\label{quantum_gravimetry_signal}
\Delta \Phi = 4 \frac{\mu_{B} \eta }{\hbar} g \bigg(\frac{\tau}{\Delta W}\bigg)^{3} = 4 \frac{\mu_{B} \eta }{\hbar} g t^{3},
\end{equation}

were we have used the definition of the non-dimensional time ($\tau\equiv \Delta W t$). Thus, we see that in the quantum picture, the total phase difference is the same as the phase difference obtained in the semi-classical computation Eq. \ref{gravimetry_signal}, except for a factor of two, and the presence of the Bohr magneton $\mu_{B}$ instead of the expectation value of the magnetic moment $<\mu_{z}>$. In addition, it can be seen that the total phase difference does not depend on the position $x_{d}$ as expected.

\paragraph{First-Order Correction to the Gravimetry Signal}
In this section, we will analyze the contribution of the rest of the terms in Eq. \ref{quantum_phase}, which we re-write here

\begin{multline*}\label{quantum_phase_revisited}
\varphi(x, \tau) = -\tau \bigg(q_{1} + \frac{\epsilon}{3} q_{2}^{2} \tau^{2}\bigg) + \left[\frac{\sigma_{x}(0)}{\sigma_{x}(\tau)} \right]^{2} (\rho_{0} - q_{2} \tau)(x-\rho_{0} \epsilon \tau) \\
+ \frac{\epsilon \tau}{4 [\sigma_{x}(\tau)\sigma_{x}(0)]^{2}} \left[(x-x_{0})^{2} + 4\rho_{0} x_{0} \epsilon \tau -2q_{2} \epsilon \tau^{2} \bigg(x+x_{0}- \frac{q_{2}}{2} \epsilon \tau^{2} \bigg)\right] 
- \frac{1}{2}\arctan\left[\frac{\epsilon \tau}{(\sigma_{x}(0))^{2}}\right]
\end{multline*}.

In the computation of the last section, we ignored the following terms

\begin{equation*}
\frac{\epsilon \tau}{4 [\sigma_{x}(\tau)\sigma_{x}(0)]^{2}} \left[(x-x_{0})^{2} + 4\rho_{0} x_{0} \epsilon \tau -2q_{2} \epsilon \tau^{2} \bigg(x+x_{0}- \frac{q_{2}}{2} \epsilon \tau^{2} \bigg)\right] 
- \frac{1}{2}\arctan\left[\frac{\epsilon \tau}{(\sigma_{x}(0))^{2}}\right]
\end{equation*}

We want to know what is the contribution of these terms to the total phase difference and get a correction for the result obtained in Eq. \ref{quantum_gravimetry_signal}. Let us begin using Eq. \ref{xi_defs_eq_motion} to re-write the above terms as 

\begin{equation}\label{quantum_phase_ignored_terms}
\frac{\epsilon \tau}{4 [\sigma_{x}(\tau)\sigma_{x}(0)]^{2}} \left[(x-x_{0})^{2} + 2v_{0} x_{0} \tau +a \tau^{2} \bigg(x+x_{0}+ \frac{a}{4} \tau^{2} \bigg)\right] 
- \frac{1}{2}\arctan\left[\frac{\epsilon \tau}{(\sigma_{x}(0))^{2}}\right]
\end{equation}

We can notice in the above equation that the last term does not have any dependence on the initial conditions (position and/or velocity) nor the acceleration. Therefore, since the initial conditions will have a dependence on $a$ (and consequently a dependence on $q_{2}$) for the second and third intervals, the contribution of the last term to the final phase difference will be trivially zero. Then, we can safely ignore the last term since its contribution will be zero and only consider the following terms:

\begin{equation}\label{quantum_phase_ignored_terms_simplified}
\frac{\epsilon \tau}{4 [\sigma_{x}(\tau)\sigma_{x}(0)]^{2}} \left[(x-x_{0})^{2} + 2v_{0} x_{0} \tau +a \tau^{2} \bigg(x+x_{0}+ \frac{a}{4} \tau^{2} \bigg)\right].
\end{equation}

Now, we can expand the first term inside the square brackets and use the same argument as before to ignore the $x^{2}$ term. Thereby, it only remains to study the contribution to the signal of the following terms

\begin{equation}\label{quantum_phase_ignored_terms_simplified_expanded}
\frac{\epsilon \tau}{4 [\sigma_{x}(\tau)\sigma_{x}(0)]^{2}} \left[-2x x_{0}+x_{0}^{2} + 2v_{0} x_{0} \tau +a \tau^{2} \bigg(x+x_{0}+ \frac{a}{4} \tau^{2} \bigg)\right].
\end{equation}

The contribution of these terms to the final signal is not trivial. We will study its contribution to the final result using the Eq. \ref{quantum_phase_ignored_terms_simplified}  because it is easier to manipulate. We will repeat the computation of the last section, however, this time, we will not use the approximation in Eq. \ref{sigma_zero_approx} Therefore, at the end of the first interval, the phase will be given by

\begin{multline}\label{quantum_phase_ignored_terms_simplified_t1}
\varphi_{\pm, \tau_{1}}(x) =-\tau \bigg(q_{1} + \frac{\epsilon}{3} q_{2_{\pm}}^{2} \tau^{2}\bigg) + \bigg[\frac{\sigma_{x, \tau_{0}}}{\sigma_{x, \tau_{1}}}\bigg]^{2}(\rho_{0} - q_{2_{\pm}} \tau)(x-\frac{v_{0} \tau}{2}) \\
+ \frac{\epsilon \tau}{4 [\sigma_{x, \tau_{1}}\sigma_{x, \tau_{0}}]^{2}} \left[(x-x_{0})^{2} + 2v_{0} x_{0} \tau +a_{\pm} \tau^{2} \bigg(x+x_{0}+ \frac{a_{\pm}}{4} \tau^{2} \bigg)\right],
\end{multline}

where we have used, as usual, a sub-index in $a$ to indicate the sign used in the definition of $q_{2}$, and the non-dimensional standard deviation $\sigma_{x, \tau_{1}}$ is given by 

\begin{equation}\label{sigma_x_t1}
\sigma_{x, \tau_{1}} = \sqrt{[\sigma_{x, \tau_{0}}]^{2} + \left[\frac{\epsilon \tau}{\sigma_{x, \tau_{0}}} \right]^{2}},
\end{equation}

where $\sigma_{x, \tau_{0}}$ is the initial standard deviation of the wave-packet.

For the second interval, the initial conditions will have changed such that the phase will be given by

\begin{multline}\label{quantum_phase_ignored_terms_simplified_t2}
\varphi_{\pm, \tau_{2}}(x) = -(2\tau) \bigg(q_{1} + \frac{\epsilon}{3} q_{2_{\mp}}^{2} (2\tau)^{2}\bigg) + \bigg[\frac{\sigma_{x, \tau_{1}}}{\sigma_{x, \tau_{2}}}\bigg]^{2}((\rho_{0} - q_{2_{\pm}} \tau)-q_{2_{\mp}} (2\tau))(x-\frac{(v_{0}+a_{\pm}\tau) (2\tau)}{2})\\ 
+\frac{\epsilon (2\tau)}{4 [\sigma_{x, \tau_{2}}\sigma_{x, \tau_{1}}]^{2}} \bigg[\bigg(x-(x_{0}+v_{0} \tau + \frac{a_{\pm}\tau^{2}}{2})\bigg)^{2}
+ 2(v_{0} + a_{\pm} \tau) (x_{0}+v_{0} \tau + \frac{a_{\pm}\tau^{2}}{2}) (2\tau) \\ 
+a_{\mp} (2\tau)^{2} \bigg(x+(x_{0}+v_{0} \tau + \frac{a_{\pm}\tau^{2}}{2})+ \frac{a_{\mp}}{4} (2\tau)^{2} \bigg)\bigg],
\end{multline}

where the new standard deviation $\sigma_{x, \tau_{2}}$ will be given in terms of the new initial standard deviation $\sigma_{x, \tau_{1}}$ as follows

\begin{equation}\label{sigma_x_t2}
\sigma_{x, \tau_{2}} = \sqrt{[\sigma_{x, \tau_{1}}]^{2} + \left[\frac{\epsilon (2\tau)}{\sigma_{x, \tau_{1}}} \right]^{2}}
\end{equation}

Similarly, for the last interval we will have

\begin{multline}\label{quantum_phase_ignored_terms_simplified_t3}
\varphi_{\pm, \tau_{3}}(x) = -\tau \bigg(q_{1} + \frac{\epsilon}{3} q_{2_{\pm}}^{2} \tau^{2}\bigg) + \bigg[\frac{\sigma_{x, \tau_{2}}}{\sigma_{x, \tau_{3}}}\bigg]^{2}((\rho_{0} - q_{2_{\pm}} \tau-2q_{2_{\mp}}\tau)-q_{2_{\pm}} \tau)(x-\frac{(v_{0}+a_{\pm}\tau + 2a_{\mp}\tau) \tau}{2}) \\
+ \frac{\epsilon \tau}{4 [\sigma_{x, \tau_{3}}\sigma_{x, \tau_{2}}]^{2}} \bigg[\bigg(x-\Big((x_{0}+v_{0} \tau + \frac{a_{\pm}\tau^{2}}{2}) + (v_{0}+a_{\pm}\tau)(2\tau) + \frac{a_{\mp}(2\tau)^{2}}{2}\Big)\bigg)^{2}\\
+ 2\bigg(v_{0} + a_{\pm} \tau + 2a_{\mp} \tau\bigg)\bigg((x_{0}+v_{0} \tau + \frac{a_{\pm}\tau^{2}}{2}) + (v_{0}+a_{\pm}\tau)(2\tau) + \frac{a_{\mp}(2\tau)^{2}}{2}\bigg) \tau\\
+a_{\pm} \tau^{2} \bigg(x+\Big((x_{0}+v_{0} \tau + \frac{a_{\pm}\tau^{2}}{2}) + (v_{0}+a_{\pm}\tau)(2\tau) + \frac{a_{\mp}(2\tau)^{2}}{2}\Big)+ \frac{a_{\pm}}{4} \tau^{2} \bigg)\bigg],
\end{multline}

where the new standard deviation $\sigma_{x, \tau_{3}}$ will be given by 

\begin{equation}\label{sigma_x_t3}
\sigma_{x, \tau_{3}} = \sqrt{[\sigma_{x, \tau_{2}}]^{2} + \left[\frac{\epsilon \tau}{\sigma_{x, \tau_{2}}} \right]^{2}}.
\end{equation}

Before proceeding with the computation of the correction for the gravimetry signal, we will give an interpretation of the approximation used in the last section. In order to do so, we need to write the square inverse of $\sigma_{x}(\tau)$ as follows

\begin{equation}\label{sigma_x_beta}
[\sigma_{x}(\tau)]^{-2} = [\sigma_{x}(0)]^{-2}\left[1 + \beta^{2}\right]^{-1},
\end{equation}

where we have defined the parameter $\beta$ as follows

\begin{equation}\label{beta_parameter}
\beta \equiv \frac{\epsilon \tau}{[\sigma_{x}(0)]^{2}}.
\end{equation}

By using a Maclaurin expansion for Eq. \ref{sigma_x_beta} in terms of $\beta$, we can see that the approximation used in the last section (Eq. \ref{sigma_zero_approx}) was just the zeroth-order expansion of $1/\sigma_{x}(\tau)$ in powers of $\beta$, i.e.,

\begin{equation}
\sigma_{x}(\tau) \approx \sigma_{x}(0).
\end{equation}

This approximation means that the width of the wave packet remains constant and is independent of the time. Thus, the width of the wave packet will be the same during all the intervals. i.e., 

\begin{equation}
\sigma_{x, \tau_{0}} \approx \sigma_{x, \tau_{1}} \approx \sigma_{x, \tau_{2}} \approx \sigma_{x, \tau_{3}} 
\end{equation}

Thereby, the approximation in the last section (Eq. \ref{quantum_gravimetry_signal}) was obtained using the zeroth-order approximation of Eq. \ref{sigma_x_beta}.
\vfill

Now, we return to the computation of correction of the gravimetry signal. At the beginning of the experiment, the initial width of the wave packet is given by $\sigma_{x, \tau_{0}}$. The next correction to the zeroth-order approximation is obtained using the first-order expansion of $\beta$. Thus we can consider the final width (Eq. \ref{sigma_x_t1}) to be given at first-order by

\begin{equation}\label{sigma_x_t1_first_order}
[\sigma_{x, \tau_{1}}]^{-2} \approx [\sigma_{x, \tau_{0}}]^{-2}\left[1 - \bigg(\frac{\epsilon \tau}{[\sigma_{x, \tau_{0}}]^{2}}\bigg)^{2}\right] = [\sigma_{x, \tau_{0}}]^{-2}\left[1 - \beta^{2}\right],
\end{equation}

Likewise, for the second interval, the final width (Eq.  \ref{sigma_x_t2}) will be given at first-order by

\begin{equation}\label{sigma_x_t2_first_order}
[\sigma_{x, \tau_{2}}]^{-2} \approx [\sigma_{x, \tau_{0}}]^{-2}\left[1 - \bigg(\frac{\epsilon (\tau+2\tau)}{[\sigma_{x, \tau_{0}}]^{2}}\bigg)^{2}\right] = [\sigma_{x, \tau_{0}}]^{-2}\left[1 - 9\beta^{2}\right],
\end{equation}

In the same way, the width at the end of the third interval (Eq. \ref{sigma_x_t3}) will be given at first-order by 

\begin{equation}\label{sigma_x_t3_first_order}
[\sigma_{x, \tau_{3}}]^{-2} \approx [\sigma_{x, \tau_{0}}]^{-2}\left[1 - \bigg(\frac{\epsilon (\tau+2\tau+\tau)}{[\sigma_{x, \tau_{0}}]^{2}}\bigg)^{2}\right] = [\sigma_{x, \tau_{0}}]^{-2}\left[1 - 16\beta^{2}\right].
\end{equation}

Finally, by using Eqs. \ref{quantum_phase_ignored_terms_simplified_t1}, \ref{quantum_phase_ignored_terms_simplified_t2}, and \ref{quantum_phase_ignored_terms_simplified_t3} alongside the first-order approximations for the width of the wave packet (Eqs. \ref{sigma_x_t1_first_order}, \ref{sigma_x_t2_first_order}, and \ref{sigma_x_t3_first_order}), we can get the total phase difference accumulated by the wave packet in the first-order approximation

\begin{comment}
\begin{multline}
\Delta \Phi = \frac{\tau }{12 \left(1 -10 \beta ^2 + 9 \beta ^4\right)} 
\Bigg[ 24 \tau ^2 \epsilon \text{} \left( \text{$(q_{2_{+}})$}^2- \text{$(q_{2_{-}})$}^2\right) \\
+ \beta ^2 \bigg(6 \left(5 x+3 x_0\right) (\text{$q_{2_{-}}$}-\text{$q_{2_{+}}$})+12 \rho _0 \tau  \epsilon  (\text{$q_{2_{-}}$}-\text{$q_{2_{+}}$}) + 231\tau ^2 \epsilon  \text{} \left( \text{$(q_{2_{-}})$}^2- \text{$(q_{2_{+}})$}^2\right)\bigg) \\
+ \beta ^4 \bigg(6 \left(65 x-51 x_0\right) (\text{$q_{2_{-}}$}-\text{$q_{2_{+}}$})-1896 \rho _0 \tau  \epsilon  (\text{$q_{2_{-}}$}-\text{$q_{2_{+}}$}) + 1377 \tau ^2 \epsilon   \text{} \left( \text{$(q_{2_{-}})$}^2- \text{$(q_{2_{+}})$}^2\right) \bigg) \\
+ \beta ^6 \bigg(18 \left(85 x_0-393 x\right) (\text{$q_{2_{-}}$}-\text{$q_{2_{+}}$})+13440 \rho _0 \tau  \epsilon  (\text{$q_{2_{-}}$}-\text{$q_{2_{+}}$}) -4773 \tau ^2 \epsilon  \left(\text{$(q_{2_{-}})$}^2-\text{$(q_{2_{+}})$}^2\right)\bigg) \\
+ \beta ^8 \bigg(54 \left(199 x-41 x_0\right) (\text{$q_{2_{-}}$}-\text{$q_{2_{+}}$})-32940 \rho _0 \tau  \epsilon  (\text{$q_{2_{-}}$}-\text{$q_{2_{+}}$}) + 32427 \tau ^2 \epsilon  \left(\text{$(q_{2_{-}})$}^2-\text{$(q_{2_{+}})$}^2\right)\bigg) \\
+ \beta ^{10} \bigg(-972 \left(5 x-x_0\right) (\text{$q_{2_{-}}$}-\text{$q_{2_{+}}$})+21384 \rho _0 \tau  \epsilon  (\text{$q_{2_{-}}$}-\text{$q_{2_{+}}$}) -27702 \tau ^2 \epsilon  \left(\text{$(q_{2_{-}})$}^2-\text{$(q_{2_{+}})$}^2\right)\bigg)\Bigg]
\end{multline}
\end{comment}

\begin{multline}
\Delta \Phi = \frac{\tau }{4 \left(1 -10 \beta ^2 + 9 \beta ^4\right)} 
\Bigg[ 8 \tau ^2 \epsilon \text{} \left( \text{$(q_{2_{+}})$}^2- \text{$(q_{2_{-}})$}^2\right) \\
+ \beta ^2 (\text{$q_{2_{-}}$}-\text{$q_{2_{+}}$}) \bigg(2 \left(5 x+3 x_0\right) +4 \rho _0 \tau  \epsilon + 77\tau ^2 \epsilon  \text{} (\text{$q_{2_{-}}$}+\text{$q_{2_{+}}$})\bigg) 
+ \mathcal{O}(\beta ^4) \Bigg].
\end{multline}

This result can be re-written using the definitions \ref{non_dimensional_definitions}-\ref{gamma_1} alongside the approximations \ref{kappa_approx}-\ref{gamma_1_approx} and keeping only quadratic terms in $\beta$ in the denominator as

\begin{equation}
\Delta \Phi = \frac{1}{\left(1 -10 \beta ^2 + 9 \beta ^4\right)} \frac{\mu_{B} \eta}{\hbar}
\Bigg[ 4 g t^3
- \beta ^2 \bigg(\frac{77 g t^3}{2} + \frac{p_{0}t^2}{M}  + (3z_{0}+5z)t \bigg) \Bigg],
\end{equation}

where $\beta$ was defined in Eq. \ref{beta_parameter}.
These previous results can be cast into a more illuminating form by expanding the denominator in the first term  and keeping only quadratic terms in $\beta$ as follows

\begin{equation}\label{gravimetry_signal_first_order_approx}
\Delta \Phi ^{(1)} = 
 2 \tau ^3 \epsilon \text{} \left( \text{$(q_{2_{+}})$}^2- \text{$(q_{2_{-}})$}^2\right)
- \beta ^2 \bigg[ \frac{\tau }{4} (\text{$q_{2_{+}}$}-\text{$q_{2_{-}}$}) \left(2 \left(5 x+3 x_0\right) +4 \rho _0 \tau  \epsilon - 3\tau ^2 \epsilon  \text{} (\text{$q_{2_{+}}$}+\text{$q_{2_{-}}$}) \right) \bigg],
\end{equation}

\begin{equation}\label{gravimetry_signal_first_order_approx_dimensional}
\Delta \Phi ^{(1)} =  \frac{\mu_{B} \eta}{\hbar}
\Bigg[ 4 g t^3
- \beta ^2 \bigg(-\frac{3 g t^3}{2} + \frac{p_{0}t^2}{M}  + (3z_{0}+5z)t \bigg) \Bigg],
\end{equation}

where we have used a super-index to indicate that these results correspond to the first-order correction.
The comparison between the zeroth-order approximation and the first-order correction can be seen in Fig. \ref{phase_graph_first_order}.

The correction given by the first-order approximation goes like $\beta ^2$. Therefore, the error made by considering the first-order approximation will be set by $\beta ^2$ which for the case of typical experimental values (see description of Fig. \ref{phase_graph_first_order}), has a value of $\beta ^2 \approx 3.3 \times 10^{-10}$. As we can see in the Fig. \ref{fig:gravimetry_signal_zeroth_order}, this gravimetry signal could allow us to measure $g$ with $9$ digits of precision, however, according to the Fig. \ref{fig:gravimetry_signal_first_minus_zeroth_order} the error due to the first-order correction would start affecting the eighth digit of the measurement. Therefore, in reality, we should be able to measure $g$ with $8$ digits of precision. This precision is affected principally by the initial width of the wave packet as can be seen in the definition \ref{beta_parameter}. For every order of magnitude that we increase(decrease) the initial width of the wave packet, we gain(lose) $4$ digits of precision in the measurement because the correction goes like $\beta ^2$.

\begin{figure}
     \centering
     \begin{subfigure}[b]{0.9\textwidth}
         \centering
         \includegraphics[width=\textwidth]{fase0order.png}
         \caption{Gravimetry signal in the zeroth-order approximation ($\Delta \Phi ^{(0)}$).}
         \label{fig:gravimetry_signal_zeroth_order}
     \end{subfigure}
     \hfill
     \begin{subfigure}[b]{0.9\textwidth}
         \centering
         \includegraphics[width=\textwidth]{phase0orderminus1order.png}
         \caption{Phase difference between the zeroth-order approximation ($\Delta \Phi ^{(0)}$) and the first-order approximation ($\Delta \Phi ^{(1)}$), i.e., this is the graph of the correction term (second term) in the Eq. \ref{gravimetry_signal_first_order_approx_dimensional}.}
         \label{fig:gravimetry_signal_first_minus_zeroth_order}
     \end{subfigure}
     \hfill
     \caption{Comparison of the gravimetry signal obtained by using the zeroth-order approximation ($\Delta \Phi ^{(0)}$) given by Eq. \ref{quantum_gravimetry_signal} versus the first-order correction ($\Delta \Phi ^{(1)}$) given by Eq. \ref{gravimetry_signal_first_order_approx_dimensional}. The period used to produce these graphs was $T=50ms$. Besides, we used $g=9.8m/s^{2}$, $\mu_{B}=9.27\times10^{-24} J/T$ (the Bohr magneton), $\eta=0.05 T/m$. Additionally, we consider $\ce{^{87}_{}Rb}$ atoms, so we have (\cite{Bunge1993}\cite{KAUSHALSK1970}) $M=1.45\times10^{-25}kg$, $\Delta W=2\pi\times6.835\times10^{9}s^{-1}$ ($\hbar \Delta W $ is the field-free ground state hyperfine energy splitting). Also, we use the approximations in Eqs. \ref{kappa_approx} and \ref{gamma_1_approx}. In addition, we take $p_{0}=0$, $z_{0}=0$, $\Delta Z(0)=1000\times10^{-6}m$, and take $z$ to be the final position of the center of the wave packet at the end of the experiment. The center of the wave packet was evaluated by using the Eq. \ref{xi_eq_motion}, where $x=\kappa z$. In this case, $(\sigma_{x, \tau_{0}})^{2} \approx 4.1 \times 10^{-8}$ and $\epsilon \tau \approx 1.5 \times 10^{-11} t$. Thus, for $t=50ms$, we have $\beta \approx 1.8 \times 10^{-5}$ and the expansion used in the Eqs. \ref{sigma_x_t1_first_order}-\ref{sigma_x_t3_first_order} is justified. Moreover, the gap between both approximations will be proportional to $\beta ^2 \approx 3.3 \times 10^{-10}$.}
     \label{phase_graph_first_order}
\end{figure}


\paragraph{Calculation of the Signal for Gravimetry in Momentum Space}
The result of the last section was obtained by considering a coherent state wave packet in position space (Eq. \ref{wave_packet}). However, the same result can be obtained by considering the state and its evolution in the momentum space \cite{Castanos2014}

\begin{equation}
\widehat{\phi}(k, \tau) = \widehat{\phi}(k + q_{2}\tau, 0) exp\left[-i q_{1} \tau + i \frac{\epsilon}{3q_{2}} k^{3} - i \frac{\epsilon}{3q_{2}} (k + q_{2} \tau)^{3} \right]
\end{equation}

where $k$ is the momentum of the state and the rest of the parameters were defined in the last section. The above equation gives the evolution of the state in the momentum space recursively. Therefore, by considering the same sequence of pulses as before (Eq. \ref{pulses}), the state at the end of the first interval will be given by

\begin{equation}\label{state_t1_momentum_space}
\widehat{\phi}_{\pm}(k, \tau) = \widehat{\phi}_{\pm}(k + q_{2_{\pm}}\tau, 0) exp\left[-i q_{1} \tau + i \frac{\epsilon}{3q_{2_{\pm}}} k^{3} - i \frac{\epsilon}{3q_{2_{\pm}}} (k + q_{2_{\pm}} \tau)^{3} \right]
\end{equation}

where, one more time, the sub-index indicates whether we are considering a plus/minus sign in front of $\gamma_{1}$ in the definition of $q_{2}$ (Eq. \ref{q1_q2}). In the same way, at the end of the second interval the state will be given by

\begin{equation}\label{state_t2_momentum_space}
\widehat{\phi}_{\pm}(k, \tau+2\tau) = \widehat{\phi}_{\pm}(k + 2q_{2_{\mp}}\tau, \tau) exp\left[-2i q_{1} \tau + i \frac{\epsilon}{3q_{2_{\mp}}} k^{3} - i \frac{\epsilon}{3q_{2_{\mp}}} (k + 2q_{2_{\mp}} \tau)^{3} \right]
\end{equation}

whereas at the end of the last interval, the state will be given by

\begin{equation}\label{state_t3_momentum_space}
\widehat{\phi}_{\pm}(k, \tau+2\tau+\tau) = \widehat{\phi}_{\pm}(k + q_{2_{\pm}}\tau, \tau+2\tau) exp\left[-i q_{1} \tau + i \frac{\epsilon}{3q_{2_{\pm}}} k^{3} - i \frac{\epsilon}{3q_{2_{\pm}}} (k + q_{2_{\pm}} \tau)^{3} \right].
\end{equation}

We can substitute Eq. \ref{state_t2_momentum_space} into Eq. \ref{state_t3_momentum_space} to get

\begin{multline}\label{state_t3_momentum_space_2}
\widehat{\phi}_{\pm}(k, \tau+2\tau+\tau) = \widehat{\phi}_{\pm}(k + q_{2_{\pm}}\tau + 2q_{2_{\mp}}\tau, \tau)  \\ 
exp\left[-2i q_{1} \tau + i \frac{\epsilon}{3q_{2_{\mp}}} (k+q_{2_{\pm}}\tau)^{3} - i \frac{\epsilon}{3q_{2_{\mp}}} ((k+ q_{2_{\pm}}\tau) + 2q_{2_{\mp}} \tau)^{3} \right] \\
exp\left[-i q_{1} \tau + i \frac{\epsilon}{3q_{2_{\pm}}} k^{3} - i \frac{\epsilon}{3q_{2_{\pm}}} (k + q_{2_{\pm}} \tau)^{3} \right].
\end{multline}

Similarly, we can substitute Eq. \ref{state_t1_momentum_space} into the last equation to get

\begin{multline}\label{state_t3_momentum_space_final}
\widehat{\phi}_{\pm}(k, \tau+2\tau+\tau) = \widehat{\phi}_{\pm}(k + q_{2_{\pm}}\tau + 2q_{2_{\mp}}\tau + q_{2_{\pm}}\tau , 0) \\ 
exp\left[-i q_{1} \tau + i \frac{\epsilon}{3q_{2_{\pm}}} (k+ q_{2_{\pm}}\tau + 2q_{2_{\mp}}\tau)^{3} - i \frac{\epsilon}{3q_{2_{\pm}}} ((k+q_{2_{\pm}}\tau + 2q_{2_{\mp}}\tau) + q_{2_{\pm}} \tau)^{3} \right] \\
exp\left[-2i q_{1} \tau + i \frac{\epsilon}{3q_{2_{\mp}}} (k+q_{2_{\pm}}\tau)^{3} - i \frac{\epsilon}{3q_{2_{\mp}}} ((k+ q_{2_{\pm}}\tau) + 2q_{2_{\mp}} \tau)^{3} \right] \\
exp\left[-i q_{1} \tau + i \frac{\epsilon}{3q_{2_{\pm}}} k^{3} - i \frac{\epsilon}{3q_{2_{\pm}}} (k + q_{2_{\pm}} \tau)^{3} \right].
\end{multline}

This equation describes each of the two states of the superposition at the end of the last interval and can be used to compute the total phase difference between them. Note that the first term on the right side of this equation can be interpreted as a wave packet that has not evolved in time and in consequence, it does not contribute to the  phase difference as can be easily proved using an initial wave packet as in Eq. \ref{wave_packet} and using the Fourier transform to write it in momentum space. For this reason, the phase gained for each state will be given by

\begin{multline}\label{phase_momentum_space}
\widehat{\varphi}_{\pm}(k) = -4q_{1}\tau \\
+ \frac{\epsilon}{3q_{2_{\pm}}} \left[(k+ q_{2_{\pm}}\tau + 2q_{2_{\mp}}\tau)^{3} - ((k+q_{2_{\pm}}\tau + 2q_{2_{\mp}}\tau) + q_{2_{\pm}} \tau)^{3} + k^{3} - (k+q_{2_{\pm}}\tau)^{3} \right] \\
+ \frac{\epsilon}{3q_{2_{\mp}}} \left[(k+q_{2_{\pm}}\tau)^{3} - ((k+q_{2_{\pm}}\tau)+2q_{2_{\mp}}\tau)^{3} \right].
\end{multline}

Thereby, the total phase difference between both states when the last $\frac{\pi}{2}$-pulse is applied will be given by

\begin{equation}
\Delta \Phi = \widehat{\varphi}_{+}(k) - \widehat{\varphi}_{-}(k) = 2\left[(q_{2_{+}})^{2} - (q_{2_{-}})^{2} \right]\epsilon \tau^{3} .
\end{equation}

Thus, by using Eqs. \ref{epsilon}-\ref{q1_q2} and the approximations in Eqs. \ref{kappa_approx} and \ref{gamma_1_approx}, we finally get the total phase difference

\begin{equation}\label{quantum_gravimetry_signal_momentum_space}
\Delta \Phi = 4 \frac{\mu_{B} \eta }{\hbar} g \bigg(\frac{\tau}{\Delta W}\bigg)^{3} = 4 \frac{\mu_{B} \eta }{\hbar} g t^{3},
\end{equation}

where we have used $\tau\equiv \Delta W t$. We can notice that this result is independent of the momentum $k$ and that it coincides with the result obtained in position space as expected.

\subsection{A Revision to the Time-Evolution of the Free Wave Packet}
In this section we review the time-evolution of a free Gaussian wave packet. Then, we use this result to establish a correspondence with some terms of the wave packet of alkali-metal atoms subject to a constant gravitational field and interacting with a magnetic field given by Eq. \ref{wave_packet_non_dimensional_evolution}.

We begin by considering a superposition of plane waves with different amplitudes, i.e.,

\begin{equation}\label{free_wave_packet_momentum_space_integral}
    \Psi (z, t) =  \int_{- \infty}^{\infty} dk A(k) e^{i (kz-\omega t)} ,
\end{equation}

where $k$ is the wave number and the amplitude $A(k)$ is of the Gaussian form

\begin{equation}\label{gaussian_amplitude_free_wave_packet}
    A(k) = C e^{-\alpha(k-k_{0})^{2}/2} ,
\end{equation}

where $k_{0}$ is the center of the distribution, $C$ is a scale factor that we can use to normalize the wave packet, and $\alpha$ is a parameter that defines the initial width of the wave packet as we will see in a moment. In order to compute the integral in the Eq. \ref{free_wave_packet_momentum_space_integral}, we need to establish the dependence between $\omega$ and $k$, i.e., the dispersion relation. We will consider the dispersion relation to be sharply peaked about $k=k_{0}$ and use a Taylor expansion about this point to second order, i.e.,

\begin{equation}\label{dispersion_relation_free_wave_packet}
    \omega(k) = \omega_{0} + (k-k_{0})\upsilon_{g} + \frac{1}{2}(k-k_{0})^{2} \beta ,
\end{equation}

where $\omega_{0}=\omega(k_{0})$, and $\upsilon_{g}$ is the group velocity defined as

\begin{equation}
    \upsilon_{g} \equiv \bigg(\frac{\partial \omega(k)}{\partial k}\bigg)_{k=k_{0}},
\end{equation}

and $\beta$ is given by

\begin{equation}
    \beta \equiv \bigg(\frac{\partial^2 \omega(k)}{\partial k^2}\bigg)_{k=k_{0}}.
\end{equation}

Now, we can substitute the Eqs. \ref{gaussian_amplitude_free_wave_packet}-\ref{dispersion_relation_free_wave_packet} into the Eq. \ref{free_wave_packet_momentum_space_integral} and workout the integral. Then, the wave packet in position space, normalized and initially centered around $z_{0}$, is given by

\begin{equation}
    \Psi (z, t) = \bigg(\frac{\sqrt{\alpha}}{2 \pi ^{3/2}} \bigg)^{1/2} \sqrt{\frac{2 \pi}{\alpha + i \beta t}} \exp \left[{\frac{i}{2}\frac{(z - z_{0} - \upsilon_{g} t)^{2}}{\beta t - i \alpha}} \right]  \exp[{i (k_{0}(z-z_{0})-\omega_{0} t)}].
\end{equation}

We can cast the above equation into a more illuminating form by putting the complex factors in the denominators in the standard form ($Z = [\Re(Z)+i\Im(Z)]$), and by using the de Moivre's theorem to compute the required squared root. In this way, we obtain the expansion of the free Gaussian wave packet as follows

\begin{multline}\label{free_wave_packet_position_space_centered_0} 
    \Psi (z, t) = (2\pi)^{-1/4} \bigg(\frac{\alpha^{2} + \beta^{2} t^{2}}{2\alpha}\bigg)^{-1/4} \exp \left[-\frac{1}{4} \frac{ (z - z_{0} - \upsilon_{g} t)^{2}}{(\alpha^{2} + \beta^{2}t^{2})/(2\alpha)}  \right] \\ \exp \left[i \bigg(k_{0}(z-z_{0}) - w_{0}t + \frac{\beta t}{2}\frac{(z - z_{0} - \upsilon_{g} t)^{2}}{\alpha^{2} + \beta^{2}t^{2}} - \frac{1}{2} \arctan\bigg[\frac{\beta t}{\alpha}\bigg]\bigg) \right],
\end{multline}

If we re-accommodate terms, we can write the equation describing the evolution of the normalized free wave packet as

\begin{multline}\label{free_wave_packet_position_space_centered_z0_final_form}
    \Psi (z, t) = \left[\frac{1}{2 \pi \big(\sigma_{z}(t)\big)^2} \right]^{1/4} \exp \left[-\frac{1}{4} \frac{ \big(z - z_{0} - \upsilon_{g} t \big)^{2}}{\big(\sigma_{z}(t)\big)^{2}} \right] \exp \left[-\frac{i}{2} \arctan\Bigg(\frac{\beta t }{2\big(\sigma_{z}(0)\big)^{2}}\Bigg) \right] \\ \exp \left[i \bigg(k_{0}(z-z_{0}) - w_{0}t + \frac{\big[(z - z_{0})^{2} + 2 z_{0} \upsilon_{g} t \big]}{ 4[\sigma_{z}(t)\sigma_{z}(0)]^{2}} \frac{\beta t}{2} + \frac{\upsilon_{g} t (\upsilon_{g} t - 2z)}{4[\sigma_{z}(t)\sigma_{z}(0)]^{2}} \frac{\beta t}{2} \bigg) \right],
\end{multline}

where we have defined the standard deviation of the wave packet as

\begin{equation}
\sigma_{z}(t) = \sqrt{\frac{\alpha^{2} + \beta^{2}t^{2}}{2 \alpha}}.
\end{equation}

Thereby, the initial standard deviation of the free wave packet is given by

\begin{equation}
\sigma_{z}(0) = \sqrt{\alpha / 2}.
\end{equation}

Hence, we can re-write the standard deviation as

\begin{equation}\label{free_wave_packet_width} 
\sigma_{z}(t) = \sqrt{[\sigma_{z}(0)]^{2} + \left[\frac{\beta t}{2 \sigma_{z}(0)} \right]^{2}}.
\end{equation}

Now, we will re-write the wave packet of alkali atoms (Eq. \ref{wave_packet_non_dimensional_evolution}) but this time, in order to establish the connection with the free wave packet, we will ignore the terms that include $q_{2}$ since this parameter is related with the acceleration of the wave packet (see Eq. \ref{xi_defs_eq_motion}) and for that reason those terms does not have an analogue in the free wave packet. Therefore, the wave packet of alkali atoms (ignoring terms that include $q_{2}$) will be given by

\begin{multline}\label{alkali_wave_packet_position_space_final_form} 
    \phi (x, \tau) = \left[\frac{1}{2 \pi \big(\sigma_{x}(\tau)\big)^2} \right]^{1/4} \exp \left[-\frac{1}{4} \frac{ \big(x - x_{0} - 2\rho_{0} \epsilon \tau \big)^{2}}{\big(\sigma_{x}(\tau)\big)^{2}} \right] \exp \left[-\frac{i}{2} \arctan\Bigg(\frac{\epsilon \tau}{\big(\sigma_{x}(0)\big)^{2}}\Bigg) \right] \\ \exp \Bigg[i \bigg(\left[\frac{\sigma_{x}(0)}{\sigma_{x}(\tau)} \right]^{2} \rho_{0}x - q_{1}\tau + \frac{\big[(x-x_{0})^{2} + 4 x_{0} \rho_{0} \epsilon \tau \big]}{ 4[\sigma_{x}(\tau)\sigma_{x}(0)]^{2}} \epsilon \tau - \left[\frac{\sigma_{x}(0)}{\sigma_{x}(\tau)} \right]^{2} \rho_{0}^{2} \epsilon \tau \Bigg],
\end{multline}

where the standard deviation of the wave packet is given by

\begin{equation}\label{alkali_wave_packet_width} 
\sigma_{x}(\tau) = \sqrt{[\sigma_{x}(0)]^{2} + \left[\frac{\epsilon \tau}{\sigma_{x}(0)} \right]^{2}}.
\end{equation}

Next, we rewrite here the definitions of the non-dimensional quantities that appear in the alkali wave packet

\begin{center}
\begin{tabular}{||c ||} 
 \hline
 non-dimensional definitions \\ [0.5ex] 
 \hline\hline
 $\sigma_{x}(0) = \kappa \Delta Z(0)$ \\ 
 \hline
 $\rho_{0} = p_{0} / \hbar \kappa$\\
 \hline
 $x = \kappa z$\\
 \hline
 $\tau = \Delta W t$\\
 \hline
 $q_{1} = 1/2$\\
 \hline
 $\epsilon = \frac{\hbar \kappa^{2}}{2 M \Delta W}$\\ [1ex] 
 \hline
\end{tabular}
\end{center}

Therefore, by comparing the Eq. \ref{free_wave_packet_width} with the Eq. \ref{alkali_wave_packet_width}, and the first row of Eq. \ref{free_wave_packet_position_space_centered_z0_final_form} with the first row of Eq. \ref{alkali_wave_packet_position_space_final_form}, we can establish the following correspondence between the terms of the free wave packet and the terms of the wave packet of alkali atoms

\begin{center}
\begin{tabular}{||c | c||} 
 \hline
 Free wave packet & Alkali wave packet \\ [0.5ex] 
 \hline\hline
 $\sigma_{z}(t) = \sqrt{\frac{\alpha^{2} + \beta^{2}t^{2}}{2 \alpha}} = \sqrt{[\sigma_{z}(0)]^{2} + \left[\frac{\beta t}{2\sigma_{z}(0)} \right]^{2}}$ & $\sigma_{x}(\tau) = \sqrt{[\sigma_{x}(0)]^{2} + \left[\frac{\epsilon \tau}{\sigma_{x}(0)} \right]^{2}}$ \\ 
 \hline
 $\sigma_{z}(0) = \sqrt{\alpha / 2}$ & $\sigma_{x}(0) (\frac{1}{\kappa}) = \Delta Z(0)$ \\
 \hline
 $\frac{\beta}{2} = \Big(\frac{\partial^2 \omega(k)}{\partial k^2}\Big)_{k=k_{0}} = \frac{\hbar}{2M}$ & $\epsilon (\frac{\Delta W}{\kappa^{2}}) = \frac{\hbar}{2 M}$ \\
 \hline
 $\upsilon_{g} = \Big(\frac{\partial \omega(k)}{\partial k}\Big)_{k=k_{0}} = \frac{\hbar k_{0}}{M}$ & $2 \rho_{0} \epsilon (\frac{\Delta W}{\kappa}) = \frac{p_{0}}{M} = \frac{\hbar k_{0}}{M}$ \\ [1ex] 
 \hline
\end{tabular}
\end{center}

Observe that $2 \rho_{0} \epsilon$ is related with the velocity of the center of the wave packet as we already shown in the Eq. \ref{xi_defs_eq_motion}.

On the other hand, by comparing the second row of Eq. \ref{free_wave_packet_position_space_centered_z0_final_form} and the second row of Eq. \ref{alkali_wave_packet_position_space_final_form}, we can see that there exists the following correspondence

\begin{center}
\begin{tabular}{||c | c||} 
 \hline
 Free wave packet & Alkali wave packet \\ [0.5ex] 
 \hline\hline
 $\omega_{0} = \omega(k_{0}) = \hbar k_{0}^{2} / 2M$ & $\Delta W q_{1} = \Delta W /2$ \\ [1ex] 
 \hline
\end{tabular}
\end{center}

Finally, we can see that the term $\rho_{0}$ seems to be related with the initial momentum $k_{0}$, i.e.

\begin{center}
\begin{tabular}{||c | c||} 
 \hline
 Free wave packet & Alkali wave packet \\ [0.5ex] 
 \hline\hline
 $k_{0}z$ & $\left[\frac{\sigma_{x}(0)}{\sigma_{x}(\tau)} \right]^{2} \rho_{0}x = \left[\frac{\sigma_{x}(0)}{\sigma_{x}(\tau)} \right]^{2} \rho_{0} \kappa z =  \left[\frac{\sigma_{x}(0)}{\sigma_{x}(\tau)} \right]^{2} \frac{p_{0}}{\hbar} z$ \\ [1ex] 
 \hline
\end{tabular}
\end{center}

This is not surprising since we already knew that $\rho_{0}=p_{0}/\hbar \kappa$. Although the correspondence is not direct due to the presence of the expansion term in the alkali wave packet.

Meanwhile for the rest of terms, we can see that some of them are present in one wave packet but they are not present in the other and vice-versa while other terms look similar in both wave packets but they differ in some way that is not possible to establish a direct connection between them. Incidentally, the terms of the second row in both Eqs. \ref{free_wave_packet_position_space_centered_z0_final_form} and \ref{alkali_wave_packet_position_space_final_form} are the terms that contribute to the phase while the terms in the first rows are those terms that does not contribute to the phase.

\subsection{Selection of the Atomic Wave Function's Width in Position Space}
We demonstrated in Eq. \ref{quantum_gravimetry_signal_momentum_space}, that it is possible to measure $g$ from the phase difference at the port of the proposed interferometer. In an ideal case, we could apply the Rabi pulses using the correct Rabi frequencies to drive the desired transitions. However, in reality, the detuning will not be zero. Furthermore, it will depend on the position of the atom, as we will show in a moment. Thus, we need to establish the conditions within which our result is valid and the gravimetry signal does not depend on the initial position of the atom.
In order to do so, first, we will establish the theoretical framework to describe the initialization of the atom to be used during the interferometry procedure. Specifically, we will show that inducing Rabi transitions in the hyperfine levels of an atom placed inside an inhomogeneous magnetic field causes its position wave function to have an specific width when the initial width is larger than the selection range set by the power of the microwaves. Then, we will calculate the initial width of the atoms just before measurement process by considering a cloud of alkali atoms in thermal equilibrium. Finally, in the next sections, we will use these results to choose the best possible initial width.

\subsubsection{Measurement of the Atom's Position Using Microwaves}
In this section, we will show that applying a microwave pulse to induce transitions in the hyperfine states of an alkali atom placed inside an inhomogeneous magnetic field causes a collapse in the width of its position wave function as long as the initial width was larger than the selection width set by the power of the microwaves used to drive the oscillations. 

We start by remembering that the experiment takes place inside a magnetic gradient field on the z-axis of the following form

\begin{equation*}
\textbf{B} = \eta z \hat{\textbf{z}},
\end{equation*}

where $\eta$ is a constant.

We will consider transitions between two hyperfine states levels of the atom with different magnetic quantum number.\footnote{For example, we can choose the ground state to be the state with total angular momentum number $F=0$ and magnetic number $m_{F}=0$ while the excited state could be that with $F=1$ and $m_{F}=1$, i.e.,

\begin{equation}
\begin{aligned}
\ket{g} = \ket{F=0, m_{F}=0} \\
\ket{e} = \ket{F=1, m_{F}=1}
\end{aligned}
\end{equation}

The reader can imagine that we are working with these states. Nonetheless, the calculation is general and can be used with the above states or any other pair of states constituting a hyperfine two level system, e.g., stretch states with $m_{F}=\pm F$.}

\begin{equation}\label{hyperfine_levels}
\begin{aligned}
\ket{g} = \ket{F_{g}, m_{F, g}} \\
\ket{e} = \ket{F_{e}, m_{F, e}}
\end{aligned}
\end{equation}

where $m_{F, e}$($m_{F, g}$) is the magnetic quantum number of the excited(ground) state, and $F_{g}$($F_{e}$) is the total angular momentum of the excited(ground) state.
The transitions in this two-level system are characterized by a frequency  $\omega_{21}$, i.e.,

\begin{equation}
  \omega_{21} = \frac{E_{e}-E_{g}}{\hbar}.
\end{equation}

Since the atom is placed inside this external magnetic field, its energy levels will be split according to the (linear) Zeeman effect, i.e.,

\begin{equation}
  \Delta E = \mu_{B} g_{F} m_{F} B,
\end{equation}

where $\mu_{B}$ is the Bohr magneton and $g_{F}$ is the Landé $g_{F}$-factor. Moreover, due to the linear dependence of the magnetic field with the position, the Zeeman split will depend on the position as well, i.e.,

\begin{equation}
  \Delta E(z) = \mu_{B} g_{F} m_{F} \eta z,
\end{equation}

This change in the energy due to the Zeeman effect can be translated into a change in the corresponding frequency of the energy level with magnetic number $m_{F}$ through the Planck relation, i.e.,

\begin{equation}
  \Delta \omega(z, g_{F}, m_{F}) = \frac{\mu_{B} g_{F} m_{F} \eta}{\hbar} z.
\end{equation}

We suppose that the wave packet describing the atom is initially located at $z_{0}$ where the magnetic field is $B_{0}$, and induce Rabi transitions between our two states by applying a perturbation with a frequency tuned with the transition that interests us. Then, both levels will suffer a shift, so the new transition frequency will be given by 

\begin{equation}
  \omega_{21}^{\prime} = \omega_{21} + \Delta \omega(z_{0}, g_{F_{e}}, m_{F_{e}}) -  \Delta \omega(z_{0}, g_{F_{g}}, m_{F_{g}}) =  \omega_{21} + \frac{\mu_{B} g_{F_{eff}} m_{F_{eff}} \eta}{\hbar} z_{0},
\end{equation}

where we have defined an effective product between the quantum magnetic numbers and the Landé $g_{F}$-factors as

\begin{equation}
    g_{F_{eff}} m_{F_{eff}} \equiv g_{F_{e}} m_{F_{e}} - g_{F_{g}} m_{F_{g}}.
\end{equation}

Notice that if we choose a level with $m_{F}=0$, then, this state will not have a change in energy due to the Zeeman effect, and we will simply have the product $g_{F} m_{F}$ of the other level\footnote{At least one level has to be sensible to the magnetic field.}.

Now, due to the uncertainty in the position of the atom, its wave function will have a width $\Delta Z$ in position space. Therefore, the probability of finding the atom in a position $z$, different from $z_{0}$, will not be zero, and in that case the frequency of the transition now will be given by

\begin{equation}
  \omega_{B} =  \omega_{21} + \frac{\mu_{B} g_{F_{eff}} m_{F_{eff}} \eta}{\hbar} z.
\end{equation}

Hence, the detuning of the Rabi oscillations due to the uncertainty in the position of the atom will be given by

\begin{equation}\label{detunig_atom_width}
  \delta (z) = \omega_{B} - \omega_{21}^{\prime} = \frac{\mu_{B} g_{F_{eff}} m_{F_{eff}} \eta}{\hbar} (z-z_{0}).
\end{equation}

\begin{figure}
\centering
\includegraphics[width=0.9\textwidth]{atom_wave_func_pos_space.png}
\caption{Atomic wave function in position space. The wave packet is centered about $z_{0}$ and has a width equal to $2Z_{HWHM}$ where $Z_{HWHM}$ defines the half width at half maximum of the distribution function.}
\label{atom_wave_func_pos_space}
\end{figure}

It turns out that this detuning sets a constraint in the width that the atom's wave function in position space can have after the stimulated transition has succesfully occurred. In order to see this, let's consider the wave function of the atom in position space to be a Gaussian like wave packet, initially centered at $z_{0}$. We already worked out the evolution of such wave packet in the last section. We re-write here the time-evolution of the Gaussian free wave packet

\begin{multline}\label{atom_wave_function_position_space}
    \Psi (z, t) = \left[\frac{1}{2 \pi \big(\sigma_{z}(t)\big)^2} \right]^{1/4} \exp \left[-\frac{1}{4} \frac{ \big(z - z_{0} - \upsilon_{g} t \big)^{2}}{\big(\sigma_{z}(t)\big)^{2}} \right] \exp \left[-\frac{i}{2} \arctan\Bigg(\frac{\beta t }{\big(\sigma_{z}(0)\big)^{2}}\Bigg) \right] \\ \exp \left[i \bigg(k_{0}(z-z_{0}) - w_{0}t + \frac{ \beta t (z - z_{0} - \upsilon_{g} t)^{2}}{ 4[\sigma_{z}(t)\sigma_{z}(0)]^{2}} \bigg) \right].
\end{multline}

The initial wave packet is given by

\begin{equation}\label{initial_atom_wave_function_position_space}
    \Psi (z, 0) = \left[\frac{1}{2 \pi \big(\sigma_{z}(0)\big)^2} \right]^{1/4} \exp \left[-\frac{1}{4}\bigg(\frac{z-z_{0}}{\sigma_{z}(0)}\bigg)^{2} \right] \exp \left[i \bigg(k_{0}(z-z_{0})\bigg) \right],
\end{equation}

where $\sigma_{z}(0)$ is the initial standard deviation of the distribution defined in the Eq. \ref{free_wave_packet_width}.

The complex modulus of this wave packet is a distribution for the probability of finding the atom in position space, i.e.,

\begin{equation}\label{initial_atom_wave_function_position_space_probability_distribution}
    |\Psi (z, 0)|^{2} = \frac{1}{\sigma_{z}(0) \sqrt{2 \pi}} \exp \left[-\frac{1}{2}\bigg(\frac{z-z_{0}}{\sigma_{z}(0)}\bigg)^{2} \right].
\end{equation}

We can define the half width at half maximum of this distribution and name the value of the independent variable $z$ at this point as $Z_{HWHM}$. In the Fig. \ref{atom_wave_func_pos_space}, we can see the initial probability distribution in position space (Eq. \ref{initial_atom_wave_function_position_space_probability_distribution}). Notice that we have defined the width of the atomic wave function as follows.

\begin{equation}\label{width_atomic_wave_function}
  \Delta Z = |2Z_{HWHM} - z_{0}| = (2Z_{HWHM} - z_{0}),
\end{equation}

where we have used the fact that $Z_{HWHM}$ will always be greater than $z_{0}$. Besides, it can be easily shown that the full with at half maximum for the Gaussian distribution is given by

\begin{equation}\label{FWHM_normal_distribution}
Z_{FWHM} = 2\sqrt{2ln2} \sigma_{z}(0) = 2 Z_{HWHM}.
\end{equation}

Let's assume that the atom's wave function is centered at $z_{0}$\footnote{Later, we will establish the condition that allows our method to be independent of the initial position of the atom, so we do not have to worry about the case where the atom has an initial offset from the origin.}, the point were the detuning is zero, and take it as the origin of coordinates, i.e., $z_{0}=0$. Therefore, if we evaluate the detuning at $z = 2Z_{HWHM}$, we get

\begin{equation}
  \delta (2Z_{HWHM}) = 2\frac{\mu_{B} g_{F_{eff}} m_{F_{eff}} \eta}{\hbar} Z_{HWHM}
\end{equation}

i.e.,

\begin{equation}\label{position_at_zhwhm}
  2Z_{HWHM} = \delta (2Z_{HWHM}) \frac{\hbar}{\mu_{B} g_{F_{eff}} m_{F_{eff}} \eta}.
\end{equation}

Thereby, to determine the width of our atom's wave function, we just need to determine the detuning at $z = Z_{HWHM}$. In order to do so, firstly, we recall that the probability of finding the atom in the excited state at time $t$ when it was initially in the ground state at $t=0$ is given by

\begin{equation}\label{rabi_oscillations_exact}
  |c_{e}(t)|^{2} =\frac{\Omega^{2}}{\tilde{\Omega}^{2}} \sin^{2}\bigg[\frac{\tilde{\Omega} t}{2} \bigg],
\end{equation}

where $\tilde{\Omega}$ is defined in terms of the Rabi flopping frequency $\Omega$ and the detuning $\delta$\footnote{Other references name $\tilde{\Omega}$ as the Rabi flopping frequency instead of $\Omega$. In any case, notice that $\tilde{\Omega}$ is the parameter that contains the information about the transition frequency while $\Omega$ only contains the information about the strength of the electromagnetic field.}, i.e.,

\begin{equation}\label{omega_tilde_definition}
  \tilde{\Omega} = \sqrt{\delta^{2} + \Omega^{2}}.
\end{equation}

Now, a physical argument comes into play. We need to evaluate the detuning at $2z_{HWHM}$. In order to do so, we will suppose that the detuning evaluated at $2z_{HWHM}$ is such that the probability of transition at this position  is zero, i.e.,

\begin{equation}\label{detuning_condition}
\sqrt{\big[ \delta (2z_{HWHM}) \big]^{2} + \Omega^{2}} \bigg( \frac{t}{2} \bigg) = \pi.
\end{equation}

Furthermore, we will take the probability of transition for positions $|z| \ge 2 Z_{HWHM}$ to be zero, i.e,

\begin{equation}\label{position_condition}
|c_{e}(t)|^{2} =
    \begin{cases}
        \frac{\Omega^{2}}{\tilde{\Omega}^{2}} \sin^{2}\bigg[\frac{\tilde{\Omega} t}{2} \bigg] & \text{if } |z| < 2 Z_{HWHM}\\
        0 & \text{if } |z| \ge 2 Z_{HWHM}
    \end{cases}
\end{equation}

The Eqs. \ref{detuning_condition} and \ref{position_condition} constitute the conditions of our approximation. This approximation makes sense because the detuning evaluated at positions farther away from the position that defines the width of the atom ($z=2Z_{HWHM}$) should be such that the probability of transition is always zero. In other words, we cannot induce transitions in the atom if the perturbation is applied in a position where the probability of finding the atom is zero.

Thus, we can use the Eq. \ref{detuning_condition} to determine the detuning at $z=2Z_{HWHM}$, i.e.,

\begin{equation}\label{detuning_at_zhwm}
\delta (2z_{HWHM}) = \frac{\sqrt{4 \pi^{2} - \Omega^{2} t^{2}}}{t}.
\end{equation}

In this way, the width of the atom's wave function can be computed using Eqs. \ref{width_atomic_wave_function}, \ref{position_at_zhwhm}, and \ref{detuning_at_zhwm} i.e.,

\begin{equation}\label{width_set_by_detuning}
\Delta Z = \frac{\sqrt{4 \pi^{2} - \Omega^{2} t^{2}}}{t} \frac{\hbar}{\mu_{B} g_{F_{eff}} m_{F_{eff}} \eta}.
\end{equation}

Therefore, we can use $\Omega$ and $t$ to control the width of the atom's final wave function in position space. Notice that we have not said anything about the initial width of the atom's wave function. However, if the initial width of the atom's wave function was larger than the width set by this equation, then, the final width will be given by this expression as long as the atom is found to have experimented the transition. In other words, as a result of having applied a Rabi pulse with a detuning due to the uncertainty in the position of the atom, we have caused this uncertainty to collapse into a smaller value as long as the transition had happened. Certainly, we have performed a measurement of the atom's position that has caused the wave function to collapse and its width to have a definite value. On the other hand, if the initial width was smaller than the value given by Eq. \ref{width_set_by_detuning}, then, the application of the microwave in the atom would not cause any change in the width of the wave function\footnote{We can say that the resolution of our measurement was not enough to provoke a collapse of the wave function.}. 

In the case of an atomic cloud, we can take advantage of this selector pulse to discard the atoms that did not end in the desired final state because its wave function was larger than $\Delta Z$, so we end up with a cloud of atoms with a definite uncertainty in position space.

For experimental purposes, we would like to have $\Omega$ fixed. Therefore, the selected width will be fixed by $t$. For example, if we use a $\pi$-pulse, i.e., $t=\pi/\Omega$. Then, the initial width will be given by

\begin{equation}\label{initial_width}
\Delta Z = \sqrt{3} \Omega \frac{\hbar}{\mu_{B} g_{F_{eff}} m_{F_{eff}} \eta}.
\end{equation}

On the other hand, in the case of a $\frac{\pi}{2}$-pulse, we would have

\begin{equation}\label{initial_width_pi/2}
\Delta Z = \frac{\sqrt{15}}{2} \Omega \frac{\hbar}{\mu_{B} g_{F_{eff}} m_{F_{eff}} \eta}.
\end{equation}

We remember that the electric field is several order of magnitude larger than the magnetic field, thereby it dominates, and the Rabi flopping frequency is usually taken to be

\begin{equation*}
\Omega = \frac{- e E_{0} \bra{e} r \ket{g}}{\hbar},
\end{equation*}

where $r$ is the position operator in the direction of the polarization of the electric field, $E_{0}$ is amplitude the electric field, and $e$ is the electron charge. Nonetheless, we will be considering hyperfine levels (Eq. \ref{hyperfine_levels}) for which the matrix elements of the position operator will vanish  \cite{Steck2010}. Therefore, the Rabi flopping frequency will depend solely on the magnetic field, i.e.,

\begin{equation}\label{rabi_frequency_magnetic_field}
\Omega = \frac{- B_{0} \bra{e} \mu \ket{g}}{\hbar},
\end{equation}

where $\mu$ is the total magnetic moment operator in the direction of the polarization of the magnetic field, $B_{0}$ is the amplitude of the magnetic field $\textbf{B}$ which in the case of a monochromatic polarized microwave can be written in the generic form

\begin{equation}
    \textbf{B} =  B_{0} \cos(\omega t) \boldsymbol{\hat{e}},
\end{equation}

where we have ignored the spatial variation due to the dimensions of the atom. Thereby, the amplitude of the magnetic field will set the width of the position wave function. The more intense the magnetic field, the larger the width that will be selected and vice-versa.

Finally, notice that since the initial wave function was Gaussian like, we can expect the final wave function to continue being a gaussian packet. To see this, we can write down a function that approximately satisfies the conditions in the Eqs. \ref{detuning_condition} and \ref{position_condition}. We will use one more time the Gaussian distribution, but this time the independent variable will be the detuning. We will define as before the full width at half maximum of this distribution

\begin{equation}\label{FWHM_normal_distribution}
\delta_{FWHM} = 2\sqrt{2ln2} \sigma_{\delta}(0)
\end{equation},

where $\sigma_{\delta}(0)$ is the standard deviation of the distribution.
Therefore,

\begin{equation}\label{rabi_oscillations_approx}
  |c_{e}(t)|^{2} \equiv f(\delta) = \frac{1}{f (0)} \frac{1}{\sigma_{\delta}(0) \sqrt{2 \pi}} \exp \left[-\frac{1}{2}\bigg(\frac{\delta}{\sigma_{\delta}(0)}\bigg)^{2} \right],
\end{equation}

where we have normalized the distribution such that the probability of transition when $\delta = 0$ equals $1$ for all times, i.e., $f(0)=1$. Finally, we choose the standard deviation to be equal to the detuning evaluated at  $z=2Z_{HWHM}$, i.e.,

\begin{equation}
    \sigma_{\delta}(0) = \delta(2Z_{HWHM})
\end{equation}

This approximation and the exact form of the transition probability versus the detuning can be seen in the Fig. \ref{ce2_VS_detuning}.

\begin{figure}
    \centering
    \includegraphics[width=0.9\textwidth]{ce2_VS_detuning_general.png}
     \caption{Probability of transition for the atom due to Rabi oscillations from the ground state to the excited state versus the detuning ($\Omega t$ is maintained fixed). The exact solution corresponds to the Eq. \ref{rabi_oscillations_exact} while the Gaussian approximation corresponds to the Eq. \ref{rabi_oscillations_approx}. This Gaussian function approximately satisfies the conditions used to estimate the detuning at $z=2Z_{HWHM}$ (Eqs. \ref{detuning_condition} and \ref{position_condition}) and we can expect that the final state will have a wave function with this shape.}
     \label{ce2_VS_detuning}
\end{figure}

\subsection{Selection of the Initial Width by Performing a Measurement in Position Space}
As we saw earlier, we can use a selector pulse to set the initial width of the atom's wave function in position space. This approach only works if the initial width of the wave packet is larger than the selection range set according to Eq. \ref{width_set_by_detuning}. Or seen from the perspective of Eqs. \ref{initial_width} and \ref{rabi_frequency_magnetic_field}, this approach corresponds to a low-power regime.

\subsubsection{Optimization of the Alkali Wave Packet Expansion}
In this approach to select the initial width of the atom, we release the alkali atoms from rest and they start to fall inside an inhomogeneous magnetic field. In this case, the expansion of the atom's wave function in position space is governed by Eq. \ref{alkali_wave_packet_width}. In spite that this equation was not obtained for the case of a superposition of states of the alkali atom, we will consider that it holds even in that case, so we can use it to describe the expansion of the wave packet during all the experiment. Having said that, notice that if the initial width is too small, the wave packet will expand quickly and the detuning will begin to be a problem. On the other hand, if the initial width is too large, we will not have too much time to perform the experiment. It seems that there should exist a sweet spot where the initial width is not too small or too big. We begin by rewriting the expansion of the wave packet in terms of dimensional quantities

\begin{equation}\label{final_width_dimensional}
\Delta Z (t) = \sqrt{[\Delta Z_{0}]^{2} + \left[\frac{\xi t}{\Delta Z_{0}} \right]^{2}},
\end{equation}

where we have defined the following constant

\begin{equation}\label{xi_definition_for_expansion}
    \xi \equiv \frac{\hbar}{2M}.
\end{equation}

We will try to optimize Eq. \ref{final_width_dimensional} so we can know which initial width has to be chosen to get an optimal expansion of the wave packet.

\subsubsection{Minimization of the Expansion of the Wave Packet}
Let us notice that the value of $\Delta Z_{0}$ that minimizes the expansion of the wave packet (Eq. \ref{final_width_dimensional}) is given by\footnote{In the following results, we have discarded the solutions with no physical meaning.}

\begin{equation}\label{initial_width_minimizes_final_width}
    (\Delta Z_{0})^{\ast} = \sqrt{\frac{\hbar t }{2M}}.
\end{equation}

By substituting this value into Eq. \ref{final_width_dimensional}, we get that the final width of the packet will be given by

\begin{equation}\label{final_width_minimum}
    (\Delta Z(t))_{min}  = \sqrt{\frac{\hbar t}{M}}.
\end{equation}

Therefore, the expanded wave packet will have a width $\sqrt{2}$ times larger than the initial width, i.e.,

\begin{equation}\label{ration_optimial_initial_width_vs_final_width}
    \frac{(\Delta Z_{0})^{\ast}}{(\Delta Z(t))_{min}} = \frac{1}{\sqrt{2}}.
\end{equation}

On the other hand, if we fix the initial width and try to optimize Eq. \ref{final_width_dimensional}, we will find that the optimal solution is trivially $t=0$ which is not of interest.

\subsubsection{Maximization of the Time to Reach a Given Width}
In the last section, we computed the initial width needed to minimize the final width of the wave packet when the expansion time $t$ is fixed (Eq. \ref{initial_width_minimizes_final_width}). However, as we mentioned earlier, the maximum Rabi frequency that we can obtain in the laboratory will usually set the final width that we can use to drive the transitions. Therefore, it makes sense to set the final width as a constant in Eq. \ref{final_width_dimensional} and find out the initial width that maximizes the time needed to reach the final width. Thereby, we re-write Eq. \ref{final_width_dimensional} as follows

\begin{equation}\label{expansion_wave_packet_time_solution}
    t = \frac{\Delta Z_{0}}{\xi} \sqrt{(\Delta Z_{f})^2 - (\Delta Z_{0})^2},
\end{equation}

where $\Delta Z_{f}$ is the final width of the wave packet. The initial width that maximizes this expression is given by

\begin{equation}\label{ratio_optimial_initial_width_vs_final_width_time_max}
    (\Delta Z_{0})^{\ast} = \frac{\Delta Z_{f}}{\sqrt{2}}.
\end{equation}

If we substitute back this value into Eq. \ref{expansion_wave_packet_time_solution}, we get the time that we can let the wave packet expand

\begin{equation}\label{t_maximized_given_initial_width}
    t_{max} = \frac{M}{\hbar} (\Delta Z_{f})^2.
\end{equation}

Finally, notice that if we set the initial width as fixed and try to compute the final width that maximizes Eq. \ref{expansion_wave_packet_time_solution}, the solution will be trivially an infinite final width so that case is not an interesting insight. 

\subsubsection{Summary}
We present here the results obtained in the last sections about the optimal expansion of the wave packet

\begin{center}
\begin{tabular}{||c | c | c | c | c||} 
 \hline
 \multicolumn{1}{||p{2cm}|}{\centering Dependent \\ Variable \\} & \multicolumn{1}{|p{2cm}|}{\centering Fixed \\ Value \\} & \multicolumn{1}{|p{2cm}|}{\centering Independent \\ Variable \\} & \multicolumn{1}{|p{3cm}|}{\centering Optimal \\ Value \\} & \multicolumn{1}{|p{3cm}||}{\centering Optimized \\ Dependent Variable \\} \\ [0.5ex] 
 \hline\hline
 $\Delta Z(t)$ & $t$ & $\Delta Z_{0}$ & $(\Delta Z_{0})^{\ast} = \sqrt{\frac{\hbar t }{2M}}$ & $(\Delta Z(t))_{min} = \sqrt{\frac{\hbar t }{M}}$ \\
 \hline
 $\Delta Z(t)$ & $\Delta Z_{0}$ & $t$ & $t^{\ast} = 0$ & $(\Delta Z(t))_{min} = 0$ \\
 \hline
 $t$ & $\Delta Z_{f}$ & $\Delta Z_{0}$ & $(\Delta Z_{0})^{\ast} = \frac{\Delta Z_{f}}{\sqrt{2}}$ & $t_{max} = \frac{M}{\hbar} (\Delta Z_{f})^{2}$ \\
 \hline
 $t$ & $\Delta Z_{0}$ & $\Delta Z_{f}$ & $(\Delta Z_{f})^{\ast} \longrightarrow \infty$ & $t_{max} \longrightarrow \infty$ \\ [1ex] 
 \hline
\end{tabular}
\end{center}

Lastly, we show in Fig. \ref{optimization_expansion_width} the optimization of the wave packet for the two non-trivial cases.

\begin{figure}
    \centering
    \begin{subfigure}{1\textwidth}
        \centering
        \includegraphics[width=0.8\textwidth]{minimize_expansion_width.png}
        \caption{Minimization of the final width after the expansion of the wave packet during a time $t$. The initial width $(\Delta Z_{0})^{\ast}$ that minimizes the expansion of the wave packet at any time is given in Eq. \ref{initial_width_minimizes_final_width}. The numbers on the right mean: $times$ $(\Delta Z_{0})^{\ast}$; so $0.2$ means $0.2(\Delta Z_{0})^{\ast}$, $0.3$ means $0.3(\Delta Z_{0})^{\ast}$, and so on. This plot was generated by considering $M$ in Eq. \ref{xi_definition_for_expansion} equal to the mass of the $\ce{^{87}_{}Rb}$ atom.}
        \label{minimize_expansion_width}
    \end{subfigure}
    \hfill
    \begin{subfigure}{1\textwidth}
        \centering
        \includegraphics[width=0.8\textwidth]{maximize_expansion_time.png}
        \caption{Maximization of the time needed to reach a given final width  $\Delta Z_{f}$. The initial width $(\Delta Z_{0})^{\ast}$ that maximizes the time needed to reach $\Delta Z_{f}$ is given in Eq. \ref{ratio_optimial_initial_width_vs_final_width_time_max}. The numbers on the right mean: $times$ $(\Delta Z_{0})^{\ast}$; so $0.6$ means $0.6(\Delta Z_{0})^{\ast}$, $0.7$ means $0.7(\Delta Z_{0})^{\ast}$, and so on. The dashed line represents the desired final width $\Delta Z_{f}$.}
        \label{maximize_expansion_time}
    \end{subfigure}
    \caption{Optimization of the expansion of the wave packet (Eq. \ref{final_width_dimensional}).}
    \label{optimization_expansion_width}
\end{figure}

\subsection{Selection of the Initial Width by Using an Optical Dipole Trap and the Expansion of the Atomic Cloud}
In this approach, we do not use a selection pulse, instead, we let the atomic cloud expand until the atom's wave function has the desired initial width. Therefore, we will need to compute the expansion of the atomic wave function in position space. Once we have obtained the equation that establishes the expansion of the wave packet, we will consider, as before, that it holds even in the case of a superposition of states of the alkali atom, so we can use it to describe the expansion of the wave packet during all the experiment. Afterward, we will optimize this equation to obtain the maximum time that we have to perform the experiment under this approach. Then, we will show that this approach corresponds to a high-power regime since the initial width of the atom under typical experimental conditions will result to be much smaller than the selection range set by the power of the microwaves used to drive the transition. Finally, it is important to say that in this approach, the expansion of the wave packet will depend on the temperature of the atomic cloud and the frequency of oscillation of the atoms in the initial trap.

\subsubsection{Initial Conditions of the Atomic Cloud}
The first step before we can initiate the experiment to measure $g$ is to gather the atoms to be used. Therefore, the atoms have to be initially trapped and cooled. For this purpose, let's assume that we use an optical dipole trap, and that the atoms inside this trap oscillate in the z-axis under a harmonic oscillator potential with frequency $\omega$. Furthermore, let's suppose that we keep the atoms trapped until they reach a thermal equilibrium\footnote{The conditions to get a thermal equilibrium will depend on the type of trap chosen.}. Afterward, we turn off the trap and let the cloud of atoms expand. Evidently, after we release the atoms from the trap, every atom will have a different wave function so we will have to rely on statistical methods to know the properties of the ensemble of atoms. For the next computations, we will suppose that the ensemble is a canonical ensemble kept at temperature $T$ even after we release them from the trap. This is justified because the potential is turned off instantaneously so we can expect that the thermal equilibrium stands for a short period of time since the atoms do not have enough time to reach the walls of the system to interchange kinetic energy. Besides, we suppose that we have a dilute gas where the atoms do not interact between them. Finally, it is important to mention that for the next results, we will consider that this ensemble follows the Boltzmann statistics. This is justified, since trapping atoms by simply using an optical dipole trap is not enough to produce a Bose-Einstein condensate. In order to achieve the quantum regime of Bose-Einstein statistics, we would need to additionally use evaporative cooling.

\subsubsection{Computation of the Density Matrix}
If we wish to compute statistic properties for our ensemble, we need to compute the density matrix. The density matrix for non degenerate states is defined as

\begin{equation}
    \rho = \sum_{n} w_{n} \ket{\phi_n} \bra{\phi_n},
\end{equation}

where the probability $w_{n}$ is normalized such that $\sum_{n} w_{n} = 1$. In the case where $\ket{\phi_n}$ are the eigenvectors of the Hamiltonian $H$, i.e., $H \ket{\phi_n} = E_n \ket{\phi_n}$, we have that

\begin{equation}
    w_{n} = Q^{-1} e^{-\beta E_{n}},
\end{equation}

where $w_{n}$ is the probability of finding, in the ensemble, a system with energy $E_{n}$, i.e., in the state $\ket{\phi_n}$.
Also, $Q = \sum_{n} e^{-\beta E_{n}}$ is the well known partition function, and $\beta = \frac{1}{k T}$, where k is the Boltzmann constant and $T$ is the temperature of every system in the ensemble. Therefore, we have that

\begin{equation}
    \begin{split}
        \rho & = Q^{-1} \sum_{n} e^{-\beta H} \ket{\phi_n} \bra{\phi_n} = Q^{-1} e^{-\beta H} \sum_{n} \ket{\phi_n} \bra{\phi_n} \\ & = \frac{e^{-\beta H}}{Tr(e^{-\beta H})} \\ & = Q^{-1} e^{-\beta H}.
    \end{split}
\end{equation}

Now, we define the non-normalized density matrix $\rho_u$ given by

\begin{equation}
    \rho_u \equiv e^{-\beta H}.
\end{equation}

If we take the derivative of this equation with respect to $\beta$, we have

\begin{equation}\label{bloch_eq}
    - \frac{\partial \rho_u}{\partial \beta} = H \rho_u.
\end{equation}

Thereby, we can solve this PDF to find $\rho_u$. The initial condition needed to solve this equation is that if we take the limit $\beta \longrightarrow 0$, i.e., $T \longrightarrow \infty$, we have to recover the density matrix of a completely random ensemble, that is

\begin{equation}\label{bloch_eq_initial_cond}
    \rho_u(0) = 1.
\end{equation}

From now on, we will omit the sub-index and just write $\rho$ for the non-normalized density matrix. What's more, to write the matrix elements in position space, we will use the following notation

\begin{equation}
    \bra{x} \rho \ket{x'} \equiv \rho(x, x' | \beta) = \sum_{n} Q^{-1} e^{\beta E_{n}} \phi_n (x) \phi_{n}^{*} (x').
\end{equation}

Therefore, in position space, the Eqs. \ref{bloch_eq} and \ref{bloch_eq_initial_cond} have the following form

\begin{equation}\label{bloch_eq_pos_space}
    - \frac{\partial \rho(x, x' | \beta)}{\partial \beta} = H_{x} \rho(x, x' | \beta),
\end{equation}

and

\begin{equation}\label{bloch_eq_initial_cond_pos_space}
    \rho(x, x' | 0) = \delta(x-x')
\end{equation}

Note that $H_{x}$ operates in the $x$-direction.

\subsubsection{The Density Matrix for the 1-D Free Particle}
Now, we will show how to solve Eqs. \ref{bloch_eq_pos_space} and \ref{bloch_eq_initial_cond_pos_space} for the case of the 1-dimensional free particle Hamiltonian, i.e.,

\begin{equation}
    H = \frac{\hat{p}^{2}}{2m},
\end{equation}

where $\hat{p}$ is the momentum operator given by

\begin{equation}
    \hat{p} = -i\hbar \nabla.
\end{equation}

Therefore, the equations to solve are

\begin{equation}
    - \frac{\partial \rho(x, x' | \beta)}{\partial \beta} = -\frac{\hbar^{2}}{2m} \frac{\partial^{2}}{\partial x^{2}} \rho(x, x' | \beta),
\end{equation}

and

\begin{equation}
    \rho(x, x' | 0) = \delta(x-x').
\end{equation}

The solution that satisfies these equations can be easily found to be

\begin{equation}
    \rho(x, x' | \beta) = \sqrt{\frac{m}{2\pi \hbar^{2} \beta}} \exp\bigg[-\bigg(\frac{m}{2 \hbar^{2} \beta}\bigg)(x-x')^{2}\bigg].
\end{equation}

This is the equation of a normal distribution and can be written as

\begin{equation}\label{density_matrix_free_particle}
    \rho(x, x' | \beta) = \frac{1}{\sigma \sqrt{2\pi}}\exp\bigg[-\frac{1}{2} \frac{(x-x')^{2}}{\sigma^{2}} \bigg],
\end{equation}

where the standard deviation of the distribution is given by

\begin{equation}\label{free_particle_density_mat}
    \sigma = \hbar \sqrt{\frac{\beta}{m}}.
\end{equation}

This result means that the matrix elements of the density operator in position space follow a normal distribution centered at the diagonal. Then, the diagonal elements are given by

\begin{equation}\label{free_particle_density_mat_diagonal}
    \rho(x, x | \beta) = \frac{1}{\sigma \sqrt{2\pi}}
\end{equation}

Observe that this solution is not normalizable. This makes sense since the wave function of a free particle is not normalizable either.

\subsubsection{The Density Matrix for the Harmonic Oscillator}
Now, we will solve Eqs. \ref{bloch_eq_pos_space} and \ref{bloch_eq_initial_cond_pos_space} for the case of an harmonic oscillator Hamiltonian, i.e.,

\begin{equation}
    H = \frac{\hat{p}^{2}}{2m} + \frac{m \omega^{2} \hat{x}^{2}}{2}.
\end{equation}

With this Hamiltonian, the equation to be solved becomes

\begin{equation}\label{oscillator_pde}
    -\frac{\partial \rho}{\partial \theta} = -\frac{\partial^{2} \rho}{\partial \xi^{2}} + \xi^{2}\rho,
\end{equation}

where we have introduced the non-dimensional quantities

\begin{equation}
    \xi \equiv \sqrt{\frac{m \omega}{\hbar}} x,
\end{equation}

and

\begin{equation}
    \theta \equiv \frac{\hbar \omega}{2} \beta.
\end{equation}

Similarly, the initial condition can be written using the non-dimensional variable $\xi$ by using the composition property of the Dirac delta function

\begin{equation}
    \delta[f(x)] = \sum_{i}\frac{1}{|f'(x_{i})|} \delta(x-x_{i}),
\end{equation}

where $f(x_{i})=0$ and $f'(x_i)\neq0$. Thus, if we write

\begin{equation}
    f(x)=\xi-\xi'=\sqrt{\frac{m\omega}{\hbar}}(x-x'),
\end{equation}

we get

\begin{equation}
    \delta[f(x)] = \delta(\xi-\xi')=\sqrt{\frac{\hbar}{m\omega}}\delta(x-x'),
\end{equation}

and the initial condition can be written as

\begin{equation}\label{oscillator_pde_initial_condition}
    \rho(\xi,\xi'|0) = \sqrt{\frac{m\omega}{\hbar}} \delta(\xi-\xi').
\end{equation}

In order to solve Eq. \ref{oscillator_pde}, we will analyze the solution in the high temperature limit, i.e., when $\theta \longrightarrow 0$. In this limit, the particles have to behave like a free particle so the solution has to be the Gaussian distribution in Eq. \ref{density_matrix_free_particle}. Thus, we suppose a solution of the form

\begin{equation}\label{oscillator_pre_solution}
    \rho = exp\big[-\big(a(\theta)\xi^{2} + b(\theta)\xi + c(\theta)\big)\big],
\end{equation}

where $a$, $b$, and $c$ are parameters that depend on $\theta$. If we differentiate this equation and substitute back into Eq. \ref{oscillator_pde}, we reduce the PDE into the following system of ordinary differential equations

\begin{equation}\label{oscillator_system_ode}
    \begin{split}
        \frac{d a}{d\theta} = 1 - 4a^{2} \\ \frac{d b}{d\theta}=-4ab \\ \frac{dc}{d\theta}=2a-b^{2}.
    \end{split}
\end{equation}

The first of these equations has two possible solutions,

\begin{equation}
    a = \frac{1}{2} \coth\big(2(\theta - c_{1})\big),
\end{equation}

and 

\begin{equation}
    a = \frac{1}{2} \tanh\big(2(\theta - c_{1})\big),
\end{equation}

where $c_{1}$ is a constant.
However, if we want to satisfy the initial condition, we have to choose the first solution. Then, we substitute this solution back into the second line of Eq. \ref{oscillator_system_ode}, and solve the resulting ODE to get

\begin{equation}
    b = c_{2} \coth \big( 2(\theta - c_{1}) \big) \sech \big( 2(\theta - c_{1}) \big),
\end{equation}

where $c_{2}$ is a constant. Similarly, if we substitute this solution and the solution for $a$ into the last line of Eq. \ref{oscillator_system_ode}, and solve the resulting ODE, we get

\begin{equation}
    c = \frac{1}{2} \ln \Big(\sinh \big( 2(\theta - c_{1}) \big) \Big) + \frac{c_{2}^{2}}{2} \coth \big( 2(\theta - c_{1}) \big) - \ln (c_{3}),
\end{equation}

where $c_{3}$ is a constant. Then, we can substitute back the solutions for Eq. \ref{oscillator_system_ode} into Eq. \ref{oscillator_pre_solution} to get

\begin{equation}
    \rho = \frac{c_{3}}{\sqrt{\sinh(2\theta)}} \exp \bigg[ -\coth \big(2(\theta - c_{1})) \bigg( \frac{\xi^{2}}{2} + c_{2} \sech \big( 2(\theta - c_{1}) \big)\xi + \frac{c_{2}^{2}}{2} \bigg) \bigg]. 
\end{equation}

Nonetheless, in order to satisfy the initial condition (Eq. \ref{oscillator_pde_initial_condition}), we have to choose $c_{1}=0$, so we have

\begin{equation}
    \rho = \frac{c_{3}}{\sqrt{\sinh(2\theta)}} \exp \bigg[ -\coth \big(2\theta) \bigg( \frac{\xi^{2}}{2} + c_{2} \sech \big( 2\theta \big)\xi + \frac{c_{2}^{2}}{2} \bigg) \bigg]. 
\end{equation}

Now, we have to analyze the behavior of our solution in the high temperature limit, i.e, $\theta \longrightarrow 0$. For that reason, we use the following Taylor expansions

\begin{equation}
    \begin{split}
        \sinh x = x + \frac{x^{3}}{16} + \cdots \\ \coth x = \frac{1}{x} + \frac{x}{3} + \cdots \\ \sech x = 1 - \frac{x^{2}}{2} + \cdots.
    \end{split}
\end{equation}

Then, if we expand at first order the trigonometric functions in the solution for $\rho$, we get

\begin{equation}
    \rho = \frac{c_{3}}{\sqrt{2\theta}} \exp \big[ -\frac{1}{4\theta}(\xi^{2} + 2c_{2}\xi + c_{2}^{2}) \big].
\end{equation}

Thus, if we choose $c_{2}=-\xi'$ to complete the square in the exponential, and $c_{3} = \sqrt{\frac{m\omega}{2\pi \hbar}}$, we recover the solution for the free particle (Eq. \ref{density_matrix_free_particle}) as expected. Thereby, the desired solution for the harmonic oscillator potential is

\begin{equation}\label{harmonic_osc_density_matrix}
    \rho(\xi, \xi'|\beta) = \sqrt{\frac{m\omega}{2\pi \hbar \sinh(2\theta)}} \exp \Bigg[ -\frac{\coth(2\theta)}{2} \big( \xi^{2} - 2\xi' \xi \sech(2\theta) + \xi'^{2} \big) \Big].
\end{equation}

Note that when $\theta \approx 0$, we recover the initial condition, i.e.,

\begin{equation}
    \rho(x, x' | \beta) \approx \sqrt{\frac{m\omega}{2\pi\hbar (2\theta)}} \exp \bigg(- \frac{m\omega}{2\hbar} \frac{(x-x')^{2}}{2\theta} \bigg) \xrightarrow{\theta \longrightarrow 0} \delta(x-x'),
\end{equation}

where we have used the fact that

\begin{equation}
    \lim_{a \to 0^{+}} \sqrt{\frac{1}{\pi a}} \exp \bigg(\frac{-(x-x_{0})^{2}}{a} \bigg) = \delta(x-x_{0}).
\end{equation}

Note that since the partition function is defined as the trace of the density operator, we will only need the diagonal elements to compute average values\footnote{It can be easily shown that the partition function is the trace of the density operator if we use the energy eigenkets. However, when two matrices are similar, their trace is the same, independently of the basis chosen, so this result is general.}. Fortunately, the diagonal elements of the density operator can be written in a very simple form by using the following trigonometric identity

\begin{equation}
    \coth (2\theta) \big[1-\sech(2\theta) \big] = \tanh\theta.
\end{equation}

Therefore, for the diagonal elements of the density operator, we have 

\begin{equation}
    \rho(x,x|\beta) = \sqrt{\frac{m\omega}{2\pi\hbar\sinh(2\theta)}} \exp \bigg( -\frac{m\omega}{\hbar}x^{2} \tanh(\theta) \bigg).
\end{equation}

Thus, we can compute the average value of the squared position as follows

\begin{equation}
    \overline{x^{2}} = \frac{\int x^{2} \rho(x,x|\beta) dx}{\int \rho(x,x|\beta) dx} = \frac{\hbar}{2m\omega} \coth\bigg( \frac{\omega\beta\hbar}{2} \bigg)
\end{equation}

Notice that we had to normalize the result because the density operator was not normalized. Similarly, for the average of the position, we have

\begin{equation}
    \overline{x} = \frac{\int x \rho(x,x|\beta) dx}{\int \rho(x,x|\beta) dx} = 0.
\end{equation}

Therefore, the dispersion of the distribution function in position space can be easily computed as follows

\begin{equation}\label{dispersion_harmonic_oscillator}
    \sigma_{x}^{2} = \overline{x^{2}} - \overline{x}^{2} = \frac{\hbar}{2m\omega} \coth \bigg( \frac{\hbar\omega}{2kT} \bigg).
\end{equation}

In the case of $\ce{^{87}_{}Rb}$ atoms, we have $m\approx1.4*10^{-25} Kg$. Besides, a typical temperature value for an atomic trap is $T\approx 1 \mu K$, while a typical frequency value is $\omega = 2\pi *10 kHz$. With these values, the width of the distribution function in position space becomes

\begin{equation}
    \sigma_{x} \approx 0.1 \mu m.
\end{equation}

\subsubsection{Time-Evolution of the Density Matrix}
The calculation of the last section was for the static case where the density matrix doe snot change in time. Nevertheless, once the atoms are released from the atomic trap, we can expect the density matrix to evolve in time under a new Hamiltonian. Thus, we need to compute the evolution of the density operator. Let's consider that at time $t_{0}$ the density operator is given by

\begin{equation}
    \rho(t_{0}) = \sum_{n} \omega_{n} \ket{n(t_{0})}\bra{n(t_{0})}.
\end{equation}

In the Schrodinger picture the state vectors evolve in time using the time-evolution operator as

\begin{equation}
    U(t, t_{0}) \ket{t_{0}} = \ket{t},
\end{equation}

where

\begin{equation}
    U(t,t_{0}) = \exp \bigg( -i \frac{H}{\hbar} (t-t_{0}) \bigg).
\end{equation}

Then, the density operator at time $t$ is given by

\begin{equation}
    \begin{split}
        \rho(t) = U \rho(t_{0}) U^{\dagger} = U \big( \sum_{n} \omega_{n} \ket{n(t_{0})}\bra{n(t_{0})} \big) U^{\dagger} \\ = \sum_{n} \omega_{n} U \ket{n(t_{0})}\bra{n(t_{0})} U^{\dagger}. 
    \end{split}
\end{equation}

As shown in many Quantum mechanics' textbooks, it can be easily proved from this equation that the density operator evolves according to the following equation

\begin{equation}
    i\hbar \frac{\partial \rho(t)}{\partial t} = - \big[\rho(t), H \big].
\end{equation}

However, if we could compute $U \ket{n(t_{0})}$, our problem would be solved without the need to solve this PDE. It turns out that if we use as base, the energy eingenkets of the Hamiltonian

\begin{equation}
    H \ket{n} = E_{n} \ket{n}.
\end{equation}

Then, the matrix elements of the density operator can be easily computed when the Hamiltonian does not depend on time, i.e.,

\begin{equation}\label{time_evol_density_h_independent_time}
   \begin{split}
       \bra{n} \rho(t) \ket{n'} = \bra{n} e^{-i\frac{E_{n}}{\hbar} t} \rho(t_{0}) e^{i\frac{E_{n'}}{\hbar} t} \ket{n'} \\ = e^{-i \omega_{n,n'} t} \bra{n} \rho(t_{0}) \ket{n'},
   \end{split}
\end{equation}

where we have defined

\begin{equation}
    \omega_{n,n'} = \frac{E_{n}-E_{n'}}{\hbar}.
\end{equation}

Note that this result means that when we use the energy eigenkets as base, and when the Hamiltonian is independent of time, then the matrix elements out of the diagonal oscillate while the elements in the diagonal does not change with time.

\subsubsection{Time-Evolution for the Density Matrix of the Harmonic Oscillator Under a Free Particle Hamiltonian}
Now that we know how to compute the time-evolution of the density matrix, let's compute the evolution of the density matrix when we release the atoms from the trap. When the atoms are trapped, the Hamiltonian is that of the harmonic oscillator. Thus, when the atoms are trapped, the density matrix of our system is given by Eq. \ref{harmonic_osc_density_matrix}. Then, we turn off the trap instantaneously, so we can suppose that the states immediately after we turn off the trap still be the same (sudden approximation). Therefore, we can suppose that at time $t_{0}$ the density matrix is still given by Eq. \ref{harmonic_osc_density_matrix}. Let's suppose that the Hamiltonian after we turn off the trap is given by the free particle Hamiltonian, i.e.,

\begin{equation*}
    H = \frac{\hat{p}^{2}}{2m}.
\end{equation*}

If we want to use Eq. \ref{time_evol_density_h_independent_time} to compute the time-evolution of the density matrix, we have to use the energy eigenkets of this Hamiltonian. Fortunately, we can use the momentum space eigenkets

\begin{equation}
    \hat{p}\ket{p} = p \ket{p},
\end{equation}

since the are energy eigenkets of the free particle Hamiltonian, i.e,

\begin{equation}
   H \ket{p} = \frac{\hat{p}^{2}}{2m} \ket{p} = \frac{p^{2}}{2m} \ket{p}.
\end{equation}

Thus, using this base, the time-evolution of the matrix elements of $\rho$ is given by

\begin{equation}\label{time_evo_energy_eigenkets}
    \rho(p,p',t|\beta) = e^{-i\omega_{p,p'}t} \rho(p,p',t_{0}\beta),
\end{equation}

where

\begin{equation}
    \rho(p,p',t_{0}) = \bra{p} \rho(t_{0}) \ket{p'},
\end{equation}

and

\begin{equation}\label{freq_omega_no_gravity}
    \omega_{p,p'} = \frac{p^{2}-p'^{2}}{2\hbar m}.
\end{equation}

Note that although we know the matrix elements in position space, now we need to know them in momentum space. Therefore, we have to change the base of the density matrix. This can be easily done by using the closure relation two times, i.e.,

\begin{equation}
    \begin{split}
        \rho(p,p'|\beta) = \bra{p} \rho \ket{p'} \\ = \int dx \braket{p|x} \bra{x} \rho \ket{p'} \\ =\int dx' \int dx \braket{p|x} \bra{x} \rho \ket{x'} \braket{x'|p'},
    \end{split}
\end{equation}

where

\begin{equation}
    \braket{x|p} = \frac{1}{\sqrt{2\pi \hbar}} e^{i px/\hbar}.
\end{equation}

Thus, we can write

\begin{equation}\label{density_mat_from_x_to_p}
    \rho(p,p'|\beta) = \frac{1}{2\pi \hbar} \int dx' \int dx \exp \bigg[ \frac{i}{\hbar} (p'x'-px) \bigg] \rho(x,x'|\beta).
\end{equation}

In the same way, we can show that the inverse transformation is given by

\begin{equation}\label{density_mat_from_p_to_x}
    \rho(x,x'|\beta) = \frac{1}{2\pi \hbar} \int dp' \int dp \exp \bigg[ \frac{i}{\hbar} (px-p'x') \bigg] \rho(p,p'|\beta).
\end{equation}

Therefore, we can compute the evolution of the density matrix as follows. First, the density matrix at time $t_{0}$ is given by Eq. \ref{harmonic_osc_density_matrix}, i.e.,

\begin{equation}
    \rho(x, x', t_{0}|\beta) = \sqrt{\frac{m\omega}{2\pi \hbar \sinh(2\theta)}} \exp \Bigg[ - \frac{m\omega}{2\hbar} \coth(2\theta) \big( x^{2} - 2x' x \sech(2\theta) + x'^{2} \big) \Big].
\end{equation}

Then, we use Eq. \ref{density_mat_from_x_to_p} to change the basis from position space to momentum space. The final result after performing the integrals is

\begin{equation}\label{harmonic_osc_density_mat_momentum_base}
    \rho(p, p', t_{0}|\beta) = \sqrt{\frac{\csch(\beta \hbar \omega)}{2\pi \hbar m \omega}} \exp \bigg[ -\frac{\big[ (p^{2}+p'^{2})\coth(\beta \hbar \omega) - 2pp'\csch(\beta \hbar \omega) \big]}{2m\hbar \omega} \bigg].
\end{equation}

Then, we can use Eq. \ref{time_evo_energy_eigenkets} to compute the time-evolution, i.e.,

\begin{multline}
    \rho(p, p', t|\beta) = \sqrt{\frac{\csch(\beta \hbar \omega)}{2\pi \hbar m \omega}} \exp \bigg[ \frac{-i(p-p')(p+p')\omega t}{2m\hbar \omega} \bigg] \\ \exp \bigg[ \frac{ -(p^{2}+p'^{2})\coth(\beta \hbar \omega) + 2pp'\csch(\beta \hbar \omega) }{2m \hbar \omega} \bigg].
\end{multline}

Finally, we can use Eq. \ref{density_mat_from_p_to_x} to return the matrix elements to the position base. The result after performing the integrals is

\begin{equation}
    \rho(x, x', t|\beta) = f(t) \exp \Bigg[ mw \frac{\big[ i\omega t(x^{2}-x'^{2})-(x^{2}+x'^{2})\coth(\beta \hbar \omega) + 2xx' \csch(\beta \hbar \omega) \big]}{2(\hbar + \omega^{2}t^{2}\hbar)} \Bigg],
\end{equation}

where we have defined the ampltiude $f(t)$ as follows

\begin{equation}
    f(t)=\frac{1}{2\pi \hbar} \sqrt{ \csch(\beta \hbar \omega) \frac{\big(-2\pi m t \omega^{2} \hbar + 2i\pi m \omega \hbar \coth(\beta \hbar \omega)\big)}{\big( i\omega t + \coth(\beta \hbar \omega) \big) \big( i + i\omega^{2} t^{2} \big)}}.
\end{equation}

Therefore, the matrix elements at the diagonal are

\begin{equation}
    \rho(x, x, t|\beta) = f(t) \exp \bigg[\frac{x^{2}mw \big( -\coth(\beta \hbar \omega) + \csch(\beta \hbar \omega) \big)}{\hbar (1+w^{2}t^{2})} \bigg].
\end{equation}

We can use this result to compute the average value of the squared position, i.e.,

\begin{equation}
    \overline{x^{2}} (t)= \frac{\int x^{2} \rho(x,x,t|\beta) dx}{\int \rho(x,x,t|\beta) dx} =  \frac{\hbar}{2m\omega}(1+\omega^{2}t^{2})\coth \bigg( \frac{\hbar \omega}{2 k T} \bigg).
\end{equation}

Similarly, the average value of the position is

\begin{equation}
    \overline{x^{2}} (t)= \frac{\int x \rho(x,x,t|\beta) dx}{\int \rho(x,x,t|\beta) dx} =  0.
\end{equation}

Finally, we can use these results to compute the dispersion of the distribution in position space, i.e.,

\begin{equation}\label{atomic_dispersion_no_gravity}
    \sigma_{x}^{2} (t)= \overline{x^{2}}(t) - \overline{x}^{2}(t) = \frac{\hbar}{2m\omega}(1+\omega^{2}t^{2})\coth\bigg( \frac{\hbar \omega}{2kT} \bigg).
\end{equation}

The above equation gives the expansion of the atomic distribution in position space for the atoms after they are released from the trap. Note how the expansion depends on the parameters of the trap ($\omega$ and $T$). Besides, for $t=0$ we recover the standard deviation for the harmonic oscillator distribution in position space as expected (Eq. \ref{dispersion_harmonic_oscillator}).

\subsubsection{Time-Evolution for the Density Matrix of the Harmonic Oscillator Under a Free Fall Hamiltonian}
In the previous section, we showed how the atomic distribution in position space expands with time. However, that calculation did not include the gravitational potential. In this section, we will show how to include the potential energy due to gravity in the calculation. We will show that the result in Eq. \ref{atomic_dispersion_no_gravity} holds even if we consider the gravitational potential with the difference that now distribution travels following a free fall trajectory.

In this case, when we turn off instantaneously the atomic trap at $t=t_{0}$, the new Hamiltonian will be given by\footnote{Note that the gravitational potential $U_{g}=mgx$ is the result of taking the negative of the gradient of the gravitational force $F_{g}=-mg$. This means that the acceleration is $a=-g$. Thus, in this convention $g$ is positive. Besides, we have that $x=x_{0}+v_{0}t-\frac{1}{2}gt^{2}$, therefore, the system of coordinates points upwards such that if the atom is falling, its momentum will be negative as we will see later.}

\begin{equation}\label{free_fall_hamiltonian}
    H = \frac{\hat{p}^{2}}{2m} + mg\hat{x}.
\end{equation}

We will assume that the momentum operator acts in the x-direction, i.e., $\hat{p}_{x}$, but we will continue writing just $\hat{p}$ for simplicity.
We remember that the time-evolution of the matrix elements of the density operator are given by

\begin{equation}
    \bra{n} \rho (t) \ket{n'} = \bra{n} U^{\dagger} \rho (t_{0}) U \ket{n'},
\end{equation}

and this time, the evolution operator will be given by

\begin{equation}
    U = \exp \bigg[-\frac{i}{\hbar} \bigg( \frac{\hat{p}^{2}}{2m} + mg\hat{x} \bigg) t \bigg].
\end{equation}

Last time, it was possible to compute the matrix elements of $\rho (t)$ by using the energy eigenkets. This time is not so trivial to know what are the eigenkets of the Hamiltonian in Eq. \ref{free_fall_hamiltonian}, but let's insist on continue using the momentum eigenkets $\ket{p}$. Thus, using this base, the matrix elements are given by

\begin{equation}
    \bra{p} \rho (t) \ket{p'} = \bra{p} U^{\dagger} \rho (t_{0}) U \ket{p'}.
\end{equation}

This means that we need to evaluate

\begin{equation}
    \ket{p(t)} = \exp \bigg[-\frac{i}{\hbar} \bigg( \frac{\hat{p}^{2}}{2m} + mg\hat{x} \bigg) t \bigg] \ket{p(t_{0})}, 
\end{equation}

where $\ket{p(t_{0})} \equiv \ket{p}$. In order to do so, we will prove that $\exp \big(\pm \frac{i}{\hbar} mg \hat{x} t \big) \ket{p}$ is an eigenfunction of the operator $\hat{p}$. We begin by using the following commutator relation

\begin{equation}
    \big[ \hat{p}, f(\hat{x}) \big] = -i \hbar \frac{\partial f}{\partial \hat{x}},
\end{equation}

where $f$ is a function of the operator $\hat{x}$. If we choose $f(\hat{x})= \exp \big(\pm \frac{i}{\hbar} mg \hat{x} t \big) $, then we have

\begin{equation}
    \bigg[ \hat{p}, \exp \bigg(\pm \frac{i}{\hbar} mg \hat{x} t \bigg) \bigg] = \pm mgt \exp \bigg(\pm \frac{i}{\hbar} mg \hat{x} t \bigg).
\end{equation}

Thus, if we multiply this equation by $\ket{p}$ from the right, we get the following relation

\begin{equation}
    \hat{p} \Bigg( \exp \bigg(\pm \frac{i}{\hbar} mg \hat{x} t \bigg) \ket{p} \Bigg) = (p\pm mgt) \Bigg( \exp \bigg(\pm \frac{i}{\hbar} mg \hat{x} t \bigg) \ket{p} \Bigg).
\end{equation}

Then, we see that $\exp \big(\pm \frac{i}{\hbar} mg \hat{x} t \big) \ket{p}$ is an eigenfunction of the momentum operator with eigenvalue $(p\pm mgt)$. In other words, the operator $\exp \big(\pm \frac{i}{\hbar} mg \hat{x} t \big)$ is the translation operator in momentum, since it translates $\ket{p}$ into $\ket{p+mgt}$\footnote{Note that the translation operator in momentum has the reversed sign in comparison with the usual translation operator in position space. In that case, a negative sign means a translation $x+dx$, and vice versa}. It is convenient to use a more succinct notation, we define

\begin{equation}
    k \equiv \frac{mg}{\hbar} t.
\end{equation}

Thus, we have

\begin{equation}
    \exp \bigg(\pm \frac{i}{\hbar} mg \hat{x} t \bigg) \equiv \exp \bigg(\pm \frac{i}{\hbar} \hbar k \hat{x} \bigg),
\end{equation}

so we can see that $\hbar k$ is the generator of translation in momentum space. Hence, we can proceed to compute

\begin{equation}
    \ket{p(t)} = \exp \bigg[-\frac{i}{\hbar} \bigg( \frac{\hat{p}^{2}}{2m} + \hbar k \hat{x} \bigg) t \bigg] \ket{p(t_{0})}.
\end{equation}

For that purpose, we will use the split operator method. This method is based on the Baker–Campbell–Hausdorff formula

\begin{equation}
    e^{A}e^{B} = e^{A+B+\frac{1}{2}[A, B] + \cdots},
\end{equation}

where $A$ and $B$ are operators. By using this formula, we can approximate\footnote{Note that if the operators $[A,B]$ in the Baker–Campbell–Hausdorff commute, then this formula reduces to the familiar exponential identity of real exponents. In the case of position and momentum, we have that $[\hat{x}_{i},\hat{p}_{j}]=i\hbar \delta_{i,j}$ so the Eq. \ref{split_operator_formula} is exact when the position and momentum operators act on orthogonal directions.} the time-evolution operator for a small time step, $\Delta t$, as the product of two independent time-evolution operators, i.e.,

\begin{equation}\label{split_operator_formula}
    \exp \bigg(-\frac{i}{\hbar} \frac{\hat{p}^{2}}{2m} \Delta t - \frac{i}{\hbar} \hbar \Delta k \hat{x} \bigg) \approx \exp \bigg( -\frac{i}{\hbar} \frac{\hat{p}^{2}}{2m} \Delta t \bigg) \exp \bigg(- \frac{i}{\hbar} \hbar \Delta k \hat{x} \bigg),
\end{equation}

where we have ignored non-linear terms in $\Delta t$, and $\Delta k = \frac{mg}{\hbar} \Delta t$. Using the above result, we can write

\begin{equation}
    \ket{p(\Delta t)} \approx \exp \bigg( -\frac{i}{\hbar} \frac{\hat{p}^{2}}{2m} \Delta t \bigg) \exp \bigg(- \frac{i}{\hbar} \hbar \Delta k \hat{x} \bigg) \ket{p}.
\end{equation}

Hence, to take a time-step $\Delta t$, we have to apply a translation in momentum, followed by the time-evolution step corresponding to the Hamiltonian of a free particle. More explicitly, for one step we have that\footnote{Note that the initial momentum can be positive if the atom was launched upwards when the trap was turned off, or negative if it was launched downwards.}

\begin{equation}
    \begin{split}
        \ket{p(\Delta t)} = \exp \bigg( -\frac{i}{\hbar} \frac{\hat{p}^{2}}{2m} \Delta t \bigg) \ket{p-\hbar \Delta k} \\ = \exp \bigg( -\frac{i}{\hbar} \frac{(p-\hbar \Delta k)^{2}}{2m} \Delta t \bigg)\ket{p-\hbar \Delta k}.
    \end{split}
\end{equation}

If we now take a second step, we have

\begin{equation}
    \begin{split}
        \ket{p(2\Delta t)} = \exp \bigg( -\frac{i}{\hbar} \frac{\hat{p}^{2}}{2m} \Delta t \bigg) \exp \bigg(- \frac{i}{\hbar} \hbar \Delta k \hat{x} \bigg) \ket{p(\Delta t)} \\ = \exp \bigg( -\frac{i}{\hbar} \frac{[(p-\hbar \Delta k)^{2}+(p-2\hbar \Delta k)^{2}]}{2m} \Delta t \bigg) \ket{p - 2 \hbar \Delta k}.
    \end{split}
\end{equation}

In the same way, taking another step, we have

\begin{equation}
    \begin{split}
        \ket{p(3\Delta t)} = \exp \bigg( -\frac{i}{\hbar} \frac{\hat{p}^{2}}{2m} \Delta t \bigg) \exp \bigg(- \frac{i}{\hbar} \hbar \Delta k \hat{x} \bigg) \ket{p(2\Delta t)} \\ = \exp \bigg( -\frac{i}{\hbar} \frac{[(p-\hbar \Delta k)^{2}+(p-2\hbar \Delta k)^{2} + (p-3\hbar \Delta k)^{2}]}{2m} \Delta t \bigg) \ket{p - 3 \hbar \Delta k}.
    \end{split}
\end{equation}

Taking $N$ steps we have that

\begin{equation}
    \ket{p(N \Delta t)} = \exp \bigg(-\frac{i}{\hbar} \frac{1}{2m} \sum_{n=1}^{N} (p- n\hbar \Delta k)^{2} \Delta t \bigg) \ket{p-N\hbar \Delta k}.
\end{equation}

We can recognize the sum in the exponential as a Riemann sum, thus, ff we now take the limit $N \longrightarrow \infty$, we have that $N \Delta t \longrightarrow t$, and $N\hbar \Delta k \longrightarrow mgt$, so we can write

\begin{equation}
    \ket{p(t)} = \exp \bigg(- \frac{i}{\hbar} \frac{1}{2m} I(t) \bigg) \ket{p-mgt},
\end{equation}

where $I(t)$ is a Riemann integral given by

\begin{equation}
    \begin{split}
        I(t) = \lim_{\Delta t \to 0} \sum_{n=1}^{N} (p- n\hbar \Delta k)^{2} \Delta t \\ = \int_{0}^{t}(p-mgt')dt' \\ = p^{2}t - pmgt^{2} + \frac{m^{2} g^{2} t^{3}}{3}.
    \end{split}
\end{equation}

Finally, we have for the evolution of the momentum eigenket

\begin{equation}
    \ket{p(t)} = \bigg( -\frac{i}{\hbar} \frac{1}{2m} \bigg(p^{2}t - pmgt^{2} + \frac{m^{2} g^{2} t^{3}}{3}\bigg) \bigg) \ket{p -mgt}
\end{equation}

Thus, we can use this result to compute the time-evolution of the matrix elements of the density operator

\begin{equation}\label{time_evo_density_op_free_fall_H}
    \begin{aligned}
        \rho(p,p',t|\beta) & = \bra{p} \rho (t) \ket{p'} \\ & = \bra{p(t)} \rho (t_{0}) \ket{p'(t)} \\ & = e^{-i\omega_{p,p'} t} \bra{p-mgt} \rho(t_{0}) \ket{p'-mgt} \\ & = e^{-i\omega_{p,p'} t} \rho (p-mgt, p'-mgt,t_{0}|\beta),
    \end{aligned}
\end{equation}

where we have defined

\begin{equation}
    \omega_{p,p'} = \frac{1}{2m\hbar} \big[ (p^{2}-p'^{2})-(p-p')mgt \big].
\end{equation}

Note that the matrix elements of the initial density operator appear translated by an amount $-mgt$, and that the frequency $\omega_{p,p'}$ has an extra term in comparison with Eq. \ref{freq_omega_no_gravity}. Thus, the procedure to get the evolution of the matrix elements of the density operator in the position basis is the same as before. First, we use Eq. \ref{density_mat_from_x_to_p} to write the initial density operator (Eq. \ref{harmonic_osc_density_matrix}) in the momentum basis. The result is given in Eq. $\ref{harmonic_osc_density_mat_momentum_base}$. Then, we use this result alongside Eq. \ref{time_evo_density_op_free_fall_H} to compute the time evolution of the density matrix in momentum space, i.e.,

\begin{multline}
    \rho(p,p',t|\beta) = e^{-i\omega_{p,p'} t} \sqrt{\frac{\csch(\beta \hbar \omega)}{2\pi \hbar m \omega}} \\ 
    \exp \bigg[ -\frac{\big[ ((p-mgt)^{2}+(p'-mgt)^{2})\coth(\beta \hbar \omega) - 2(p-mgt)(p'-mgt)\csch(\beta \hbar \omega) \big]}{2m\hbar \omega} \bigg]
\end{multline}

Finally, we use Eq. \ref{density_mat_from_p_to_x} to return the matrix elements to the position base. The final result after we perform the integrals is

\begin{multline}
    \rho(x,x',t|\beta) = \Delta \exp \bigg(i \frac{mt}{2\hbar} (x-x') \frac{\big[g(2+t^{2}\omega^{2})+\omega^{2}(x+x')\big]}{(1+ \omega^{2}t^{2})} \bigg) \\
    \exp \bigg( -\frac{\pi}{2} \Delta^{2} \Big[ \big( g^{2} t^{4} - 2g t^{2}(x+x') + 2(x^{2}+x'^{2}) \big)\cosh(\beta \hbar \omega) - (g t^{2} - 2x)(g t^{2} - 2x') \Big] \bigg),
\end{multline}

where

\begin{equation}
    \Delta = \sqrt{\frac{m\omega}{2\pi \hbar (1+ \omega^{2}t^{2}) \sinh (\beta \hbar \omega)}}.
\end{equation}

Therefore, the diagonal elements are given by

\begin{equation}
    \rho(x,x,t|\beta) = \Delta \exp \bigg( -\frac{\pi}{2} \Delta^{2} \Big[ \big( g^{2} t^{4} - 4g t^{2}x + 4x^{2} \big)\cosh(\beta \hbar \omega) - (g t^{2} - 2x)^{2} \Big] \bigg).
\end{equation}

Then, as usual, we can compute the average of the squared position and the position itself. After a lot of algebraic manipulation, we get

\begin{equation}
    \overline{x^{2}} (t)= \frac{\int x^{2} \rho(x,x,t|\beta) dx}{\int \rho(x,x,t|\beta) dx} =  \frac{g t^{2} (2+B g t^{2})}{4B},
\end{equation}

where 

\begin{equation}
    B = \frac{m \omega g t^{2}}{\hbar (1+w^{2}t^{2})} \tanh \bigg(\frac{\beta}{2} \hbar \omega\bigg),
\end{equation}

and also

\begin{equation}\label{average_position_dipole_trap}
    \overline{x} (t)= \frac{\int x \rho(x,x,t|\beta) dx}{\int \rho(x,x,t|\beta) dx} =  \frac{g t^{2}}{2}.
\end{equation}

Using these results, we can compute the standard deviation as a function of time

\begin{equation}\label{atomic_dispersion_with_gravity}
    \sigma_{x} (t) = \sqrt{\frac{\hbar (1+t^{2} \omega^{2})}{2 m \omega} \coth \bigg(\frac{\beta}{2} \hbar \omega \bigg)}.
\end{equation}

We can see that this results is the same as that in Eq. \ref{atomic_dispersion_no_gravity}. Thus, the only effect of gravity was to introduce a change in the average position. Indeed, the average position of the atom is that of a free-fall particle just as we could have expected.

\subsubsection{Optimization of the Atomic Wave Packet Expansion}
In this approach, we consider that the expansion of the wave packet is governed by Eq. \ref{atomic_dispersion_with_gravity} which we re-write here as

\begin{equation}\label{atomic_dispersion_with_gravity_2}
    \Delta Z (t) = \sqrt{\frac{\hbar (1+t^{2} \omega^{2})}{2 M \omega} \coth \bigg(\frac{\hbar \omega}{2 kT} \bigg)}.
\end{equation}

Notice that we can re-write this equation as

\begin{equation}\label{atomic_dispersion_with_gravity_3}
    \Delta Z (t) = \Delta Z(0) \sqrt{1+t^{2}\omega^{2}},
\end{equation}

where $\Delta Z(0)$ is the initial width just after dipole the trap is turned off, and is given by

\begin{equation}\label{atomic_dispersion_with_gravity_t0}
    \Delta Z(0) = \sqrt{\frac{\hbar}{2M \omega} \coth \bigg(\frac{\hbar \omega}{2kT} \bigg)}
\end{equation}

Let's use these equations to get more insight into the experimental values that we can use to get better results. Experimentally, besides of the time $t$, we can only adjust the angular frequency $\omega$ and the temperature of the trap $T$.

\subsubsection{Minimization of the Expansion of the Wave Packet}
If we fix the time $t$ and the frequency $\omega$, we can try to optimize Eq. \ref{atomic_dispersion_with_gravity_3} to get the initial width that minimizes it. The optimal initial width is trivially $(\Delta Z(0))^{*}=0$. However, according to Eq. \ref{atomic_dispersion_with_gravity_t0}, it is impossible to obtain an initial width of zero. Nonetheless, we can still calculate the minimum possible initial width if we optimize Eq. \ref{atomic_dispersion_with_gravity_t0} with regards to the temperature $T$. The initial width is minimized when the hyperbolic cotangent reaches its minimum value. However, the hyperbolic cotangent does not have a global or local minimum. Although, this function does not have a minimum, when the $\coth(x)$ function is defined over the interval $[0,\infty)$, it reaches its minimum value at infinity because $\lim_{x^{+} \to 0} \coth(x)=\infty$ and $\lim_{x \to \infty} \coth(x)=1$. Therefore, this function will have its minimum value at infinity even though it cannot be considered a global minimum in the usual sense because infinity is not a number. In our case, this point corresponds to $T= 0$. Thus, at $T=0$ the initial width reaches its minimum value and becomes

\begin{equation}
    \Delta Z(0) = \sqrt{\frac{\hbar}{2M \omega} } , \quad (T=0).
\end{equation}

Obviously, it is not possible to get such temperature experimentally. However, we can still get the behavior for low temperatures by expanding the hyperbolic cotangent around $T=0$, we get

\begin{equation}\label{delZ0_smallT}
    \Delta Z (0) \approx \sqrt{\frac{ kT}{ M \omega^{2}}  }, \quad (\hbar \omega \gg kT).
\end{equation}

Therefore, experimentally, this is the optimal initial width that minimizes Eq. \ref{atomic_dispersion_with_gravity_3} to is minimal experimental value, i.e.,

\begin{equation}\label{eq_optimal_initial_width}
    (\Delta Z (0))^{**} = \sqrt{\frac{ kT}{ M \omega^{2}} }, \quad (\hbar \omega \gg kT),
\end{equation}

where we have used a double star ($**$) symbol to remark that this is not an optimal value in the usual sense found in calculus textbooks. Indeed, it is an optimal value in an experimental sense. 

The above result is a good approximation for low temperatures since the hyperbolic cotangent approaches quickly to its asymptotic value at infinity. For instance, for $T\approx 0.1 \mu K$ and $\omega = 4\pi *10 kHz$, we have that $\hbar \omega \approx 1.3*10^{-29}$ and $2 kT \approx 0.3*10^{-29}$. Thus, the argument of the hyperbolic cotangent in Eq. \ref{atomic_dispersion_with_gravity_3} is $\frac{\hbar \omega}{2kT} \approx 4.8$, and we have that $\coth (4.8) \approx 1.00014$, which is very close to the asymptotic value of $1$. These temperature and frequency values are very close to the typical experimental values that can be obtained using a dipole trap. Then, the approximation used in Eq. \ref{delZ0_smallT} is a good approximation.

If we substitute the optimal initial width in Eq. \ref{atomic_dispersion_with_gravity_3}, we get that the minimum expansion, when $t$ and $\omega$ are maintained fixed, is given by

\begin{equation}
    \big( \Delta Z (t) \big)^{**}_{min} = \sqrt{\frac{ kT}{ M \omega^{2}} } \sqrt{1+t^{2}\omega^{2}}, \quad (\hbar \omega \gg kT) ,
\end{equation}



\begin{comment}
If we fix the time $t$ and $\omega$ in the last equation, and minimize it with regards to the temperature, we will find that the expansion is minimized when the hyperbolic cotangent reaches its minimum value, i.e., when $T= 0$. In that case, the expansion becomes

\begin{equation}
    \Delta Z (t) = \sqrt{\frac{ (1+t^{2} \omega^{2})}{ 2m \omega^{2}} }, \quad (T=0)
\end{equation}

Obviously, it is not possible to get such temperature experimentally. However, we can still get the behavior for low temperatures by expanding the hyperbolic cotangent around $T=0$, we get

\begin{equation}
    \Delta Z (t) \approx \sqrt{\frac{ (1+t^{2} \omega^{2})}{ m \omega^{2}} kT }, \quad (\hbar \omega \gg kT)
\end{equation}

Now, if we fix $T$ and try to minimize the expansion with respect to the temperature $\omega$ ....

Similarly, if we fix $T$ and try to minimize the expansion with respect to the temperature $\omega$, we will get that the expansion is minimized when the frequency is $\omega=0$. One more time, it is not possible, experimentally, to have $\omega=0$. Thus, let's analyze the behavior of the expansion for low frequencies. If we expand around $\omega=0$, we get

\begin{equation}
    \Delta Z (t) \approx \sqrt{\frac{kT}{ m \omega^{2}}} \bigg( 1 + \frac{\omega^{2}}{24 k^{2}T^{2}} \big(12 k^{2}T^{2} t^{2} + \hbar^{2}\big) \bigg), \quad (\hbar \omega \ll kT)
\end{equation}
\end{comment}

\begin{comment}
\begin{equation}
    \Delta Z (t) \approx \sqrt{\frac{kT}{ m \omega^{2}}} \Bigg( 1 + \Bigg[ \frac{\omega^{2} t^{2}}{2} + \frac{1}{24} \bigg(\frac{\hbar \omega}{k T}\bigg)^{2}\Bigg] \Bigg), \quad (\hbar \omega \ll kT)
\end{equation}
\end{comment}

\subsubsection{Maximization of Time to Reach a Given Width}
We can solve Eq. \ref{atomic_dispersion_with_gravity_3} to get the time as a function of the final width $\Delta Z_{f}$, i.e,

\begin{equation}\label{t_atomic_dispersion}
    t = \frac{1}{\omega} \sqrt{\bigg( \frac{\Delta Z_{f}}{\Delta Z_{0}} \bigg)^{2} - 1},
\end{equation}

If we try to optimize this last equation to maximize the time to reach the final width given an initial width, we will get again that the optimal initial width is trivially given by $(\Delta Z_{0})^{*} = 0$. However, as we saw in the last section, in practice, the initial width cannot be set to zero experimentally. The best we can do is to use Eq. \ref{eq_optimal_initial_width} as an approximation to the optimal initial width. If we substitute Eq. \ref{eq_optimal_initial_width} into Eq. \ref{t_atomic_dispersion}, we get

\begin{equation}\label{t_dipole_trap_optimized}
    t^{**}_{max} = \frac{\Delta Z_{f}}{\omega} \sqrt{\frac{M \omega^{2}}{kT} - \bigg(\frac{1}{\Delta Z_{f}}\bigg)^{2}}, \quad (\hbar \omega \gg kT),
\end{equation}

where the double star ($**$) indicates that this equation was optimized by using the initial width that best approximates the optimal initial width in an experimental sense.

\subsubsection{Constraint in the Maximum Time Set by the Available Space}
As we showed in Eq. \ref{average_position_dipole_trap}, the average position of the atoms follows a classical free-fall trajectory. Therefore, there exists a constraint in the maximum time that we can let the atom's wave function expand set by the maximum free-fall distance available to perform the experiment. Let $z_{max}$ be the maximum free-fall distance, thus, the maximum time of free-fall is given by

\begin{equation}
    t_{max} = \sqrt{\frac{z_{max}}{g}}.
\end{equation}

This equation sets the minimum value that $T$ can have for a given $\omega$ and $\Delta Z_{f}$, i.e., the minimum initial width that can be chosen. Let's use Eq. \ref{t_dipole_trap_optimized} to establish this condition. In the low temperature limit $(\hbar \omega \gg kT)$, the following condition must be satisfied

\begin{equation}\label{dipole_trap_temperature_length_condition}
\begin{split}
    t^{**}_{max} = & \frac{\Delta Z_{f}}{\omega} \sqrt{\frac{M \omega^{2}}{kT} - \bigg(\frac{1}{\Delta Z_{f}}\bigg)^{2}} \le t_{max} = \sqrt{\frac{z_{max}}{g}} \\
    \frac{1}{T} & \le \bigg( \frac{z_{max}}{g} + \frac{1}{\omega^{2}} \bigg) \frac{k}{M (\Delta Z_{f})^{2}}.
\end{split}
\end{equation}

This equation guarantees that the atom will have enough time to expand and reach the desired final width $\Delta Z_{f}$. This condition alongside the low-temperature condition $(\hbar \omega \gg kT)$ must be taken into account when designing the experiment.

\subsection{Selection of the Initial Width by Using a Magneto-optical Trap and a Selection Pulse}

\subsubsection{The Density Matrix for a particle in a box}
If we have an ensemble of free particles confined in a box, the matrix elements of the density operator are still given by Eqs. \ref{free_particle_density_mat} and \ref{free_particle_density_mat_diagonal}. However, this time the diagonal elements are normalizable. If the box (1-dimensional) has a length $L$, we have that the normalization constant is given by

\begin{equation}
   \begin{split}
        N & = \int_{-L/2}^{L/2} \rho(x, x | \beta) dx \\ & = \frac{1}{\sigma \sqrt{2\pi}} \int_{-L/2}^{L/2} dx \\ & = \frac{L}{\sigma \sqrt{2\pi}}
   \end{split}
\end{equation}

\subsection{Choosing the Initial Atomic Width}
In the last sections, we showed how we can select the initial atomic width of the alkali atoms to be used during the measurement process and how this packet expands in each case before the beginning of the experiment to measure $g$. In this section, we will study the constraints that choosing a particular initial atomic width will impose in the time that we can let the atomic wave packet expand without affecting the Rabi transitions due to the detuning, and thus, the gravimetry signal.

\subsubsection{Working Frequency}
As we showed in Eq. \ref{rabi_frequency_magnetic_field}, the intensity of the magnetic field used to drive the Rabi oscillations will set a selection range. This selection range defines the maximum uncertainty in position space that the wave function can have after a successful transition to the desired state has occurred. Furthermore, it also defines the range in position space within which the atom can successfully perform the desired transition. In order to guarantee that the transition happens with a high probability of success, the wave function's width has to lie inside this selection range. As the width of the wave function is reduced, the detuning decreases , and thus, the probability of success in the transition increases. For that reason, we would like to choose the smallest possible initial width, and the largest possible selection range. Let's suppose that the selection range $\Delta Z'$ is set by a frequency $\Omega'$ according to Eq. \ref{rabi_frequency_magnetic_field}.
Typically, the frequency $\Omega'$ will be set as the maximum frequency that could be obtained in the laboratory. According to Eq. \ref{rabi_frequency_magnetic_field}, the Rabi frequency will be directly proportional to the magnitude of the magnetic field of the microwaves used to drive the transition. This magnetic field can be measured by using a power meter to measure the power of the radiation according to

\begin{equation}
    P = I \int dA,
\end{equation}

where $dA$ is the differential element of area perpendicular to the direction and $I$ is the intensity of the electromagnetic wave. The power meter measures the intensity of the electric field $E$ and is given by

\begin{equation}
    I = \frac{1}{2} c \epsilon_{0} |E|^{2},
\end{equation}

where $\epsilon_{0}$ is the permittivity of free space and $c$ is the speed of light in vacuum. Therefore, the power of the microwaves can be written as

\begin{equation}
    P = \frac{1}{2} c \epsilon_{0} |E|^{2} A,
\end{equation}

where we have evaluated the area of the power meter. This equation can be re-written in terms of the magnetic field using the following relation

\begin{equation}
    E = c B = \frac{B}{\sqrt{\mu_{0} \epsilon_{0}}},
\end{equation}

where $\mu_{0}$ is the vacuum permeability. Thus, the power is given by

\begin{equation}\label{power_in_terms_of_B}
    P = \frac{c |B|^{2}}{2 \mu_{0}} A.
\end{equation}

Now, we can solve for the magnetic field in terms of the power measured, i.e.,

\begin{equation}
    |B| = \sqrt{\frac{2 \mu_{0} P}{c A}}.
\end{equation}

We can calculate the Rabi frequency produced by this magnetic field using Eq. \ref{rabi_frequency_magnetic_field}

\begin{equation*}
\Omega = \frac{- B \bra{e} \boldsymbol{\hat{e}} \cdot \boldsymbol{\mu} \ket{g}}{\hbar},
\end{equation*}

where $\boldsymbol{\hat{e}}$ is the direction of polarization of the magnetic field and $\boldsymbol{\mu}$ is the total magnetic moment of the atom given by the spin, orbital, and nuclear contributions, i.e.,\footnote{In this convention the eigenvalues of the operators $\boldsymbol{S}$, $\boldsymbol{L}$, and $\boldsymbol{I}$ have units of angular momentum, i.e., $Joules*Seconds$, and the Bohr magneton is defined as $\frac{\hbar e}{2 m_{e}}$. Besides, the Landé $g$-Factors $g_{S}$ and $g_{L}$ are positive while $g_{I}$ has a negative value. Notice that $\boldsymbol{\mu}$ has to have units of $Amperes*(Meters)^2$.}

\begin{equation}
     \boldsymbol{\mu} = \frac{\mu_{B}}{\hbar}(g_{S}\boldsymbol{S} + g_{L}\boldsymbol{L} - g_{I}\boldsymbol{I}).
\end{equation}

We will take the magnetic field to be along the z-axis (the atomic quantization axis). Thus, we can write the Rabi frequency as

\begin{equation}\label{working_freq_set_by_microwaves_power}
\Omega' = - \sqrt{\frac{2 \mu_{0} P}{c A}} \frac{\bra{e} \mu_{z} \ket{g}}{\hbar},
\end{equation}

where 

\begin{equation}
     \mu_{z} = \frac{\mu_{B}}{\hbar}(g_{S}S_{z} + g_{L}L_{z} - g_{I}I_{z}).
\end{equation}

\subsubsection{Gravimetry Signal Set by the Working Frequency (The Low-Power Regime)}
We can compute the selection range given by the Rabi frequency in Eq. \ref{working_freq_set_by_microwaves_power} using Eq. \ref{initial_width}, i.e.,

\begin{equation*}
\Delta Z' = \sqrt{3} \Omega' \frac{\hbar}{\mu_{B} g_{F_{eff}} m_{F_{eff}} \eta},
\end{equation*}

so we can write

\begin{equation}\label{maximum_selection_range}
\Delta Z' = - \sqrt{\frac{6 \mu_{0} P}{c A}} \frac{\bra{e} \mu_{z} \ket{g}}{\mu_{B} g_{F_{eff}} m_{F_{eff}} \eta}.
\end{equation}

The above equation gives the maximum selection range set by the power of the microwaves used to drive the transitions. We will decide to work always within this range. Our work range will be proportional to the maximum selection range set by the power of the microwaves according to

\begin{equation}\label{warning_range}
    \Delta Z_{f} = \gamma \Delta Z' \mathrm{,} \quad 0 < \gamma \leq 1,
\end{equation}

where the smaller the value of $\gamma$, the better the chances to drive the expected transitions. Now, we can use Eq. \ref{ratio_optimial_initial_width_vs_final_width_time_max} to compute the initial width that will maximize the time to reach this final width, i.e.,

\begin{equation}
    (\Delta Z_{0})^{\ast} = \frac{\gamma \Delta Z'}{\sqrt{2}}.
\end{equation}

If we select this initial width, the time needed to reach $\Delta Z_{f}$ will be given by Eq. \ref{t_maximized_given_initial_width}, i.e.,

\begin{equation}
    t_{max} = \frac{M}{\hbar} (\gamma \Delta Z')^2.
\end{equation}

If we substitute Eq. \ref{maximum_selection_range} into the last equation we get

\begin{equation}\label{total_experiment_time}
    t_{max} = \frac{6 M \mu_{0} P}{\hbar c A} \bigg(\frac{ \gamma  \bra{e} \mu_{z} \ket{g}}{\mu_{B} g_{F_{eff}} m_{F_{eff}} \eta}\bigg)^{2}.
\end{equation}

This equation represents the maximum time that we can let the wave packet expand in terms of the power of the microwaves and the proportionality factor $\gamma$. Next, we can use Eq. \ref{quantum_gravimetry_signal_momentum_space} to compute the gravimetry signal that we could obtain by performing an atomic interference experiment with a total duration equal to $t_{max}$, i.e., 

\begin{equation}\label{gravimetry_signal_T}
    \Delta \Phi = 4 \frac{\mu_{B} \eta }{\hbar} g T^{3},
\end{equation}

where $T$ controls the time of free evolution between the pulses according to Eq. \ref{pulses} (see Fig. \ref{phase_graph}). The total duration of the experiment $t_{max}$ is given by

\begin{equation}
    4 T + 2\tau_{\pi/2} + 2\tau_{\pi} = t_{max},
\end{equation}

where $\tau_{\pi/2}$ is the length of a $\frac{\pi}{2}$-pulse, and $\tau_{\pi}$ is the length of a $\pi$-pulse. However, the typical Rabi frequencies used in traditional atomic interferometry with Raman pulses are of order $\Omega \approx 10^5 s^{-1}$, which corresponds for a $\pi$-pulse to a time length of $\tau_{\pi} = \frac{\pi}{\Omega} \approx 30 \mu s$ \cite{Menoret2018}. In our case, the microwaves provide less power, and therefore, a smaller value for $\Omega$. However, we still can consider the time length of the pulses to be much larger as we will show in the next section. Thus, we can consider the Rabi frequencies to have an infinitesimal duration and use the following approximation

\begin{equation}\label{t_max_vs_T}
    4T \approx t_{max}.
\end{equation}

Thereby, using Eqs. \ref{total_experiment_time}, \ref{gravimetry_signal_T}, and \ref{t_max_vs_T}, we obtain

\begin{equation}\label{gravimetry_signal_irradiance}
    \Delta \Phi = b g I^3, 
\end{equation}

where we have defined the parameter

\begin{equation}
    b \equiv \frac{1}{2} \frac{ \mu_{B}}{ \hbar^{4} \eta^{5}} \bigg(\frac{3M \mu_{0}}{c} \bigg)^{3} \bigg(\frac{\bra{e} \mu_{z} \ket{g}}{\mu_{B} g_{F_{eff}} m_{F_{eff}}} \bigg)^{6} \gamma^6,
\end{equation}

and we have written the result in terms of the irradiance of the microwaves\footnote{The irradiance is a more convenient quantity because not all the power of the microwaves will be seen by the atom due to its finite dimensions.}, i.e.,

\begin{equation}
    I = \frac{P}{A}.
\end{equation}

\subsubsection{Gravimetry Signal Set by the Optical Dipole Trap (The High-Power Regime)}
If instead of initializing the initial width using a selection beam, we use the expansion of the dipole trap. We can substitute Eq. \ref{warning_range} into Eq. \ref{t_dipole_trap_optimized} to get

\begin{equation}
        t^{**}_{max} = \frac{1}{\omega} \sqrt{\frac{M \omega^{2}}{kT} \bigg( \gamma \Delta Z' \bigg)^{2} - 1}, \quad (\hbar \omega \gg kT),
\end{equation}

and by substituting Eq. \ref{maximum_selection_range} into this equation, we get

\begin{equation}
        t^{**}_{max} = \frac{1}{\omega} \sqrt{6 \frac{M \omega^{2}}{kT} \frac{\mu_{0} P}{ c A} \bigg(\frac{ \gamma  \bra{e} \mu_{z} \ket{g}}{\mu_{B} g_{F_{eff}} m_{F_{eff}} \eta}\bigg)^{2} - 1}, \quad (\hbar \omega \gg kT).
\end{equation}

Thus, using the approximation in Eq. \ref{t_max_vs_T} and the Eq. \ref{gravimetry_signal_T}, we get the gravimetry signal obtained when using the dipole trap to select the initial width, i.e.,

\begin{equation}\label{gravimetry_signal_irradiance_dipole_trap}
    \Delta \Phi = b_{d} g, 
\end{equation}

where we have defined the parameter

\begin{equation}
    b_{d} \equiv \frac{\mu_{B} \eta}{16 \hbar \omega^{3}} \bigg(6 \frac{M \omega^{2}}{kT} \frac{\mu_{0} I}{c} \bigg(\frac{ \gamma  \bra{e} \mu_{z} \ket{g}}{\mu_{B} g_{F_{eff}} m_{F_{eff}} \eta}\bigg)^{2} - 1\bigg)^{3/2}, \quad (\hbar \omega \gg kT),
\end{equation}

where $I$ is the irradiance of the microwaves.

\subsubsection{Selection of the Temperature}
We have to make sure that the condition in Eq. \ref{dipole_trap_temperature_length_condition} alongside the low-temperature condition $(\hbar \omega \gg kT)$ are satisfied. Let's find out the minimum temperature that can be used if we want to get the final width $\gamma \Delta Z'$. If we substitute Eq. \ref{maximum_selection_range} into Eq. \ref{dipole_trap_temperature_length_condition}, the condition becomes

\begin{equation}
    continuar aqui 
\end{equation}

Let's suppose that the maximum free-fall distance is $z_{max}=30cm$, and the trap frequency is $\omega = 4\pi*10kHz$. 

ver que tambien se cumple condicion baja temperature $(\hbar \omega \gg kT)$.

\subsection{Interferometer Fringes}
\subsubsection{Precision in the Measurement of g in the Low-Power Regime}
After the sequence of pulses has been applied not all the atoms in the cloud will have seen the same sequence of pulses. This is mainly because of the detuning introduced by the displacement in the position of the atoms due to the Zeeman effect. Therefore, after the sequence is applied, the normalized population of atoms in the upper state will be given by

\begin{equation}\label{fringes_equation}
    P = \frac{1}{2}(1+C\cos{\Delta \Phi}),
\end{equation}

where $C$ is the fringe contrast and $\Delta \Phi$ is given by Eq. \ref{gravimetry_signal_irradiance}. We define the signal-to-noise-ratio (SNR) as the ratio between the contrast of the fringes and the uncertainty on the slope of the fringe, i.e,

\begin{equation}
    SNR = \frac{C/2}{\sigma_{p}}, 
\end{equation}

where the uncertainty on the slope of the fringe can be calculated using Eq. \ref{fringes_equation}, i.e,

\begin{equation}
    \sigma_{p} = \frac{C}{2} |\sin{\Delta \Phi}| \sigma_{_{\Delta \Phi}}.
\end{equation}

Therefore, the SNR will be given by

\begin{equation}
    SNR = \frac{1}{|\sin{\Delta \Phi}|  \sigma_{_{\Delta \Phi}}}.
\end{equation}

At mid fringe we have $\Delta \Phi = \frac{\pi}{2}$, so the SNR will be given by

\begin{equation}\label{SNR_mid_fringe}
    SNR = \frac{1}{\sigma_{_{\Delta \Phi}}},
\end{equation}

Now, we take the product of the Eqs. \ref{gravimetry_signal_irradiance} and \ref{SNR_mid_fringe} as follows

\begin{equation}\label{sensitivity_in_phase}
    \frac{\Delta \Phi}{\sigma_{_{\Delta \Phi}}} = b g I^3 SNR.
\end{equation}

This equation is the sensitivity in the measurement of the gravimetry signal but we would like to have the sensitivity in the measurement of $g$. Thus,
we have to notice that since there exists a linear relation between $g$ and $\Delta \phi$ (Eq. \ref{gravimetry_signal_irradiance}), there also exists a linear relation between the uncertainty $\sigma_{g}$ and the uncertainty $\sigma_{_{\Delta \Phi}}$, i.e.,

\begin{equation}
    \sigma_{_{\Delta \Phi}} = b \sigma_{g} I^3,
\end{equation}

so we can write 

\begin{equation}
    \frac{\Delta \Phi}{\sigma_{_{\Delta \Phi}}} = \frac{g}{\sigma_{g}}.
\end{equation}

Thus, if we substitute this equation back into Eq. \ref{sensitivity_in_phase}, we get the sensitivity in the measurement of $g$, i.e.,

\begin{equation}\label{sensitivity_eq}
    \frac{\sigma_{g}}{g} = \frac{1}{b g I^3 SNR}.
\end{equation}

Thus, the precision in the measurement is given by

\begin{equation}\label{precision_eq}
    \sigma_{g} = \frac{1}{b I^3 SNR}.
\end{equation}

A plot of the theoretical sensitivity versus the power of the microwaves obtained using typical experimental values can be seen in Fig. \ref{precision_vs_power_figure}. It can be seen that for a microwaves' power of $1W$ and a value of $\gamma=0.2$, we could get around 7 digits of precision in a single shot. This precision can be improved even more by repeating the measurement and remembering that the precision improves as the square root of the number of measurements\cite{Bevington_Robinson_Blair_Mallinckrodt_McKay_1993}. Thus, if we want to improve the precision by two orders of magnitude, we could average $10^{4}$ measurements. If each measurement takes around $1s$, then, we would require around $3$ hours to reach the desired precision.

\begin{figure}
    \centering
    \includesvg[width=1\textwidth]{precision_vs_power_plot.svg}
    \caption{A plot of the theoretical sensitivity in the measurement of $g$ as a function of the power of the microwaves used to drive the transitions (Eq. \ref{sensitivity_eq}). The legends on the right represent the value of the parameter $\gamma$. The power is assumed to be measured in an area of $100cm^2$. The signal-to-noise-ratio is taken to be $SNR=20$. The magnetic field is considered to have a value of $\eta = 500 \frac{G}{m}$. We are considering a $\ce{^{87}_{}Rb}$ atom with mass $M=1.443*10^{-25}kg$ and supposing a transition in the $5^{2}S_{1/2}$ level between the hyperfine levels $\ket{F=1, m_{F}=1}$ and $\ket{F=2, m_{F}=1}$ with approximate Landé $g_{F}$-Factors $g_{F}=-1/2$ and $g_{F}=1/2$ respectively \cite{Steck2010}. We approximate $\bra{e} \mu_{z} \ket{g}$ as being equal to the Bohr magneton with value $\mu_{B}=9.274*10^{-24}J/T$. Finally, we take $g=9.81 m/s^2$. With all these values, the parameter $b$ in Eq. \ref{sensitivity_eq} has a value of $719.046 \frac{s^{11}}{kg^{3}m}$.}
    \label{precision_vs_power_figure}
\end{figure}

\subsubsection{Precision in the Measurement of g in the High-Power Regime}


\subsubsection{Phase Shift Chirp}
In order to determine $g$, we introduce a phase chirp $\alpha$ that will be used as a ramp parameter to map the population of atoms in the upper state for different phase shift values. 

\begin{equation}
    \Delta \Phi = (b g  - 2\pi \alpha) I^3.
\end{equation}

This phase chirp will help to fit a curve to the interferometry signal from which its statistical uncertainty could be determined. Moreover, it will help to determine the point with zero phase shift corresponding to the case when we stay on the central fringe of the interferometer, i.e.,

\begin{equation}
    b g  - 2\pi \alpha_{0} = 0
\end{equation}

From which we can derive

\begin{equation}
    g = \frac{2\pi \alpha_{0}}{b}
\end{equation}

\subsubsection{Systematic Error}
The systematic error will introduce a change in the phase shift. In that case, the condition to stay in the central fringe will be

\begin{equation}
    b g I^3 - 2\pi \alpha_{0} I^3 + \Delta \Phi_{sys}= 0,
\end{equation}

so the measurement will be given by

\begin{equation}\label{g_value_with_sys_error}
    g = \frac{2\pi \alpha_{0}}{b} - \frac{\Delta \Phi_{sys}}{b I^3}
\end{equation}

\subsubsection{Statistical Error}
The uncertainty for each $g$ measurement can be obtained using Eq. \ref{g_value_with_sys_error}, i.e.,

\begin{equation}
    \sigma_{g} = \frac{2\pi \sigma_{\alpha_{0}}}{b},
\end{equation}

where $\sigma_{\alpha_{0}}$ is the uncertainty of $\alpha_{0}$.

\subsection{Conditions to Perform the Measurement}

\subsubsection{Condition to Have a Gravimetry Signal Independent of the Initial Position of the Atom}
Now, we establish the conditions that allow our method to be independent of the initial position of the atom.
Suppose that we have successfully initialized our Alkali atom with a initial width $\Delta Z_{0}$. Once we have initialized the width of our atom, we can proceed applying the first $\frac{\pi}{2}$-pulse. However, as we saw in the last sections, the transitions will occur only if the atom's width is much less than the width $\Delta Z'$ set by the power of the microwaves used to drive the oscillations between the hyperfine states. Otherwise, the detuning will be considerably large such that the probability of transition will be almost null. According to Eq. \ref{initial_width}, if we use a Rabi frequency $\Omega'$ to drive the transitions\footnote{Eq. \ref{initial_width} was obtained by considering a $\pi$-pulse. Nonetheless, the sequence of pulses includes $\frac{\pi}{2}$-pulses at the end and beginning of the experiment. We will continue using Eq. \ref{initial_width} during all the calculations without loss of generality since in any case the order of magnitude is the same and we will always stay far from the limit of the selection range $\Delta Z'$.}, the atom will experience the transition if and only if its width is less or equal than the width selected by $\Omega'$, i.e.,

\begin{equation}\label{maximum_width_rabi_frequency}
    \Delta Z' = \sqrt{3} \Omega' \frac{\hbar}{\mu_{B} g_{F_{eff}} m_{F_{eff}} \eta}.
\end{equation}

Furthermore, if we want to guarantee a high probability of success in the transitions, we have to make sure that the initial width is much less than the selection range, i.e.,

\begin{equation}\label{condition_Dz'>>Dz0}
    \Delta Z' >> \Delta Z_{0}.
\end{equation}

According to this condition, we could think that choosing a very small width could be a good idea. However, after applying the first pulse, the atom must experience a time-evolution of free fall before we can apply the next pulse. During this period, the width of the atom will expand according to Eq. \ref{final_width_dimensional}. Thereby, if we chose a very small initial width, the final width will increase rapidly and the condition in Eq. \ref{condition_Dz'>>Dz0} will not hold for the rest of pulses. This means that we have to choose an appropriate initial width such that Eq. \ref{condition_Dz'>>Dz0} holds at any time. More specifically, immediately after all the sequence of pulses is applied, the final width of the atom $\Delta Z_{f}$ must be such that

\begin{equation}\label{condition_transition_width1}
    \Delta Z' > \Delta Z_{f} \ge \Delta Z_{0},
\end{equation}

and, in order to guarantee transitions with a high probability of success, the condition in Eq. \ref{condition_Dz'>>Dz0} must hold alongside

\begin{equation}\label{condition_transition_width2}
    \Delta Z_{f} \approx \Delta Z_{0}.
\end{equation}

The above conditions mean, that in order to have a high probability of success when inducing the transition, we have to make sure that $\Delta Z_{f}$ is still much less than $\Delta Z'$, i.e.,  $\Delta Z_{f}$ should not have increased considerably from $\Delta Z_{0}$. In the extreme case that $\Delta Z_{f} \ge \Delta Z'$, the probability of success in the transition will be so small that we will consider it to be zero. 
In summary, if we stay during all the experiment in a narrow region within the region set by $\Delta Z'$ (see Fig. \ref{selection_range_figure}), then, our method will not depend on the initial position of the atom, so we can neglect the error introduced by the detuning at least at first order.
therefore, when designing the experiment, we have to make sure that the above constraints hold at any time that we apply a pulse to induce Rabi transitions after we let the wave packet expand. For that reason, it is important to choose wisely the initial width of our wave packet so we can make sure that the transition between levels will occur always. Finally, notice that both Eqs. \ref{ration_optimial_initial_width_vs_final_width} and 
\ref{ratio_optimial_initial_width_vs_final_width_time_max} satisfied the condition in Eq. \ref{condition_transition_width2}.

\begin{figure}
    \centering
    \includegraphics[width=0.9\textwidth]{working_area.png}
     \caption{Selection range to induce Rabi transitions for an Alkali atom inside an inhomogeneous magnetic field. The atomic wave function is initialized with a width $\Delta Z_{0}$ (black curve). After a time $t$ the wave function expands according to Eq. \ref{final_width_dimensional} until it reaches a width $\Delta Z_{f}$ at the end of the sequence of pulses (orange curve). If we want the atom to experience all the sequence of pulses with a high probability of success, we have to make sure that the final width $\Delta Z_{f}$ lies inside a narrow range inside the selection range $\Delta Z'$ (see Eq. \ref{condition_transition_width1}). This selection range is set by the dashed curve's width and represents the hypothetical width that is selected by using a frequency $\Omega'$ to drive the transitions (Eq. \ref{maximum_width_rabi_frequency}). This frequency will be usually set by the maximum frequency that can be obtained experimentally (see Eq. \ref{rabi_frequency_magnetic_field}). The probability of transition for positions outside the selection range is so small that we can consider it to be zero.}
     \label{selection_range_figure}
\end{figure}

\subsubsection{Condition of Infinitesimal Rabi Pulses}
We obtained the result in Eq. \ref{gravimetry_signal_irradiance} by considering infinitesimal Rabi pulses. However, in reality, the duration of the Rabi pulses is not infinitesimal. Therefore, the previous result will hold if and only if $t_{max}$ is much larger than the duration of the Rabi pulses, i.e., $t_{max} >> \tau_{Rabi}$. Let's use a $\pi$-pulse, where $\tau_{\pi} = \frac{\pi}{\Omega}$, to exemplify this point. Thus, using Eq. \ref{total_experiment_time}, the condition is the following\footnote{Remember that $\Omega$ was defined in Eq. \ref{omega_tilde_definition}. In this case, we are considering the detuning to be zero so $\tilde{\Omega} = \Omega$.}

\begin{equation}
    \frac{6 M \mu_{0} P}{\hbar c A} \bigg(\frac{ \gamma  \bra{e} \mu_{z} \ket{g}}{\mu_{B} g_{F_{eff}} m_{F_{eff}} \eta}\bigg)^{2} >> \frac{\pi}{\Omega}.
\end{equation}

We can use Eq. \ref{power_in_terms_of_B} to write this expression in terms of the magnetic field

\begin{equation}
   |B|^{2}\frac{3 M }{\hbar} \bigg(\frac{ \gamma  \bra{e} \mu_{z} \ket{g}}{\mu_{B} g_{F_{eff}} m_{F_{eff}} \eta}\bigg)^{2} >> \frac{\pi}{\Omega}.
\end{equation}

Then, we can use Eq. \ref{rabi_frequency_magnetic_field} to write this expression in terms of $\Omega$, i.e.,

\begin{equation}
   3 \Omega^{2} \hbar M \bigg(\frac{ \gamma}{\mu_{B} g_{F_{eff}} m_{F_{eff}} \eta}\bigg)^{2} >> \frac{\pi}{\Omega}.
\end{equation}

Therefore, the condition of infinitesimal Rabi pulses has been translated into the following condition in $\Omega$

\begin{equation}
    \Omega  >> \Bigg[ \frac{\pi}{3 \hbar M} \bigg(\frac{\mu_{B} g_{F_{eff}} m_{F_{eff}} \eta}{\gamma}\bigg)^{2} \Bigg]^{1/3}.
\end{equation}

In the case of a $\ce{^{87}_{}Rb}$ atom, we can estimate the order of magnitude of the right side of this condition so we get

\begin{equation}
    \Bigg[ \frac{\pi}{3 \hbar M} \bigg(\frac{\mu_{B} g_{F_{eff}} m_{F_{eff}} \eta}{\gamma}\bigg)^{2} \Bigg]^{1/3} \approx 10^{3} Hz.
\end{equation}

Let's compare this value with an estimation of the maximum value that we can obtain in the laboratory. According to the Eq. \ref{working_freq_set_by_microwaves_power},  we have\footnote{The approximations used to get these two estimations are the same that we used in Fig. \ref{precision_vs_power_figure}.}

\begin{equation}
    \Omega' \approx 10^{4} Hz  >> 10^{3} Hz.
\end{equation}

Therefore, the value $\Omega'$ is around one order of magnitude larger than the threshold set by the condition of infinitesimal Rabi pulses. This means that the approximation that we used in Eq. \ref{t_max_vs_T} was legal. Finally, it is interesting to note that the frequency that we can obtain using microwaves is around one order of magnitude less than the frequencies obtained by using lasers to produce Raman pulses in traditional gravimetry.

\bibliographystyle{plain}
\bibliography{cites.bib}

























\begin{comment}
\section{Some examples to get started}

Your introduction goes here! Simply start writing your document and use the Recompile button to view the updated PDF preview. Examples of commonly used commands and features are listed below, to help you get started.

Once you're familiar with the editor, you can find various project setting in the Overleaf menu, accessed via the button in the very top left of the editor. To view tutorials, user guides, and further documentation, please visit our \href{https://www.overleaf.com/learn}{help library}, or head to our plans page to \href{https://www.overleaf.com/user/subscription/plans}{choose your plan}.

\subsection{How to create Sections and Subsections}

Simply use the section and subsection commands, as in this example document! With Overleaf, all the formatting and numbering is handled automatically according to the template you've chosen. If you're using Rich Text mode, you can also create new section and subsections via the buttons in the editor toolbar.

\subsection{How to include Figures}

First you have to upload the image file from your computer using the upload link in the file-tree menu. Then use the include graphics command to include it in your document. Use the figure environment and the caption command to add a number and a caption to your figure. See the code for Figure \ref{fig:frog} in this section for an example.

Note that your figure will automatically be placed in the most appropriate place for it, given the surrounding text and taking into account other figures or tables that may be close by. You can find out more about adding images to your documents in this help article on \href{https://www.overleaf.com/learn/how-to/Including_images_on_Overleaf}{including images on Overleaf}.

\begin{figure}
\centering
\includegraphics[width=0.3\textwidth]{frog.jpg}
\caption{\label{fig:frog}This frog was uploaded via the file-tree menu.}
\end{figure}

\subsection{How to add Tables}

Use the table and tabular environments for basic tables --- see Table~\ref{tab:widgets}, for example. For more information, please see this help article on \href{https://www.overleaf.com/learn/latex/tables}{tables}. 

\begin{table}
\centering
\begin{tabular}{l|r}
Item & Quantity \\\hline
Widgets & 42 \\
Gadgets & 13
\end{tabular}
\caption{\label{tab:widgets}An example table.}
\end{table}

\subsection{How to add Comments and Track Changes}

Comments can be added to your project by highlighting some text and clicking ``Add comment'' in the top right of the editor pane. To view existing comments, click on the Review menu in the toolbar above. To reply to a comment, click on the Reply button in the lower right corner of the comment. You can close the Review pane by clicking its name on the toolbar when you're done reviewing for the time being.

Track changes are available on all our \href{https://www.overleaf.com/user/subscription/plans}{premium plans}, and can be toggled on or off using the option at the top of the Review pane. Track changes allow you to keep track of every change made to the document, along with the person making the change. 

\subsection{How to add Lists}

You can make lists with automatic numbering \dots

\begin{enumerate}
\item Like this,
\item and like this.
\end{enumerate}
\dots or bullet points \dots
\begin{itemize}
\item Like this,
\item and like this.
\end{itemize}

\subsection{How to write Mathematics}

\LaTeX{} is great at typesetting mathematics. Let $X_1, X_2, \ldots, X_n$ be a sequence of independent and identically distributed random variables with $\text{E}[X_i] = \mu$ and $\text{Var}[X_i] = \sigma^2 < \infty$, and let
\[S_n = \frac{X_1 + X_2 + \cdots + X_n}{n}
      = \frac{1}{n}\sum_{i}^{n} X_i\]
denote their mean. Then as $n$ approaches infinity, the random variables $\sqrt{n}(S_n - \mu)$ converge in distribution to a normal $\mathcal{N}(0, \sigma^2)$.


\subsection{How to change the margins and paper size}

Usually the template you're using will have the page margins and paper size set correctly for that use-case. For example, if you're using a journal article template provided by the journal publisher, that template will be formatted according to their requirements. In these cases, it's best not to alter the margins directly.

If however you're using a more general template, such as this one, and would like to alter the margins, a common way to do so is via the geometry package. You can find the geometry package loaded in the preamble at the top of this example file, and if you'd like to learn more about how to adjust the settings, please visit this help article on \href{https://www.overleaf.com/learn/latex/page_size_and_margins}{page size and margins}.

\subsection{How to change the document language and spell check settings}

Overleaf supports many different languages, including multiple different languages within one document. 

To configure the document language, simply edit the option provided to the babel package in the preamble at the top of this example project. To learn more about the different options, please visit this help article on \href{https://www.overleaf.com/learn/latex/International_language_support}{international language support}.

To change the spell check language, simply open the Overleaf menu at the top left of the editor window, scroll down to the spell check setting, and adjust accordingly.

\subsection{How to add Citations and a References List}

You can simply upload a \verb|.bib| file containing your BibTeX entries, created with a tool such as JabRef. You can then cite entries from it, like this: \cite{greenwade93}. Just remember to specify a bibliography style, as well as the filename of the \verb|.bib|. You can find a \href{https://www.overleaf.com/help/97-how-to-include-a-bibliography-using-bibtex}{video tutorial here} to learn more about BibTeX.

If you have an \href{https://www.overleaf.com/user/subscription/plans}{upgraded account}, you can also import your Mendeley or Zotero library directly as a \verb|.bib| file, via the upload menu in the file-tree.

\subsection{Good luck!}

We hope you find Overleaf useful, and do take a look at our \href{https://www.overleaf.com/learn}{help library} for more tutorials and user guides! Please also let us know if you have any feedback using the Contact Us link at the bottom of the Overleaf menu --- or use the contact form at \url{https://www.overleaf.com/contact}.

\bibliographystyle{alpha}
\bibliography{sample}

\end{comment}

\end{document}